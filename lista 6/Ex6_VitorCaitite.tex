% Options for packages loaded elsewhere
\PassOptionsToPackage{unicode}{hyperref}
\PassOptionsToPackage{hyphens}{url}
%
\documentclass[
]{article}
\usepackage{lmodern}
\usepackage{amssymb,amsmath}
\usepackage{ifxetex,ifluatex}
\ifnum 0\ifxetex 1\fi\ifluatex 1\fi=0 % if pdftex
  \usepackage[T1]{fontenc}
  \usepackage[utf8]{inputenc}
  \usepackage{textcomp} % provide euro and other symbols
\else % if luatex or xetex
  \usepackage{unicode-math}
  \defaultfontfeatures{Scale=MatchLowercase}
  \defaultfontfeatures[\rmfamily]{Ligatures=TeX,Scale=1}
\fi
% Use upquote if available, for straight quotes in verbatim environments
\IfFileExists{upquote.sty}{\usepackage{upquote}}{}
\IfFileExists{microtype.sty}{% use microtype if available
  \usepackage[]{microtype}
  \UseMicrotypeSet[protrusion]{basicmath} % disable protrusion for tt fonts
}{}
\makeatletter
\@ifundefined{KOMAClassName}{% if non-KOMA class
  \IfFileExists{parskip.sty}{%
    \usepackage{parskip}
  }{% else
    \setlength{\parindent}{0pt}
    \setlength{\parskip}{6pt plus 2pt minus 1pt}}
}{% if KOMA class
  \KOMAoptions{parskip=half}}
\makeatother
\usepackage{xcolor}
\IfFileExists{xurl.sty}{\usepackage{xurl}}{} % add URL line breaks if available
\IfFileExists{bookmark.sty}{\usepackage{bookmark}}{\usepackage{hyperref}}
\hypersetup{
  pdftitle={Redes Neurais Artificiais},
  pdfauthor={Vítor Gabriel Reis Caitité - 2016111849},
  hidelinks,
  pdfcreator={LaTeX via pandoc}}
\urlstyle{same} % disable monospaced font for URLs
\usepackage[margin=1in]{geometry}
\usepackage{color}
\usepackage{fancyvrb}
\newcommand{\VerbBar}{|}
\newcommand{\VERB}{\Verb[commandchars=\\\{\}]}
\DefineVerbatimEnvironment{Highlighting}{Verbatim}{commandchars=\\\{\}}
% Add ',fontsize=\small' for more characters per line
\usepackage{framed}
\definecolor{shadecolor}{RGB}{248,248,248}
\newenvironment{Shaded}{\begin{snugshade}}{\end{snugshade}}
\newcommand{\AlertTok}[1]{\textcolor[rgb]{0.94,0.16,0.16}{#1}}
\newcommand{\AnnotationTok}[1]{\textcolor[rgb]{0.56,0.35,0.01}{\textbf{\textit{#1}}}}
\newcommand{\AttributeTok}[1]{\textcolor[rgb]{0.77,0.63,0.00}{#1}}
\newcommand{\BaseNTok}[1]{\textcolor[rgb]{0.00,0.00,0.81}{#1}}
\newcommand{\BuiltInTok}[1]{#1}
\newcommand{\CharTok}[1]{\textcolor[rgb]{0.31,0.60,0.02}{#1}}
\newcommand{\CommentTok}[1]{\textcolor[rgb]{0.56,0.35,0.01}{\textit{#1}}}
\newcommand{\CommentVarTok}[1]{\textcolor[rgb]{0.56,0.35,0.01}{\textbf{\textit{#1}}}}
\newcommand{\ConstantTok}[1]{\textcolor[rgb]{0.00,0.00,0.00}{#1}}
\newcommand{\ControlFlowTok}[1]{\textcolor[rgb]{0.13,0.29,0.53}{\textbf{#1}}}
\newcommand{\DataTypeTok}[1]{\textcolor[rgb]{0.13,0.29,0.53}{#1}}
\newcommand{\DecValTok}[1]{\textcolor[rgb]{0.00,0.00,0.81}{#1}}
\newcommand{\DocumentationTok}[1]{\textcolor[rgb]{0.56,0.35,0.01}{\textbf{\textit{#1}}}}
\newcommand{\ErrorTok}[1]{\textcolor[rgb]{0.64,0.00,0.00}{\textbf{#1}}}
\newcommand{\ExtensionTok}[1]{#1}
\newcommand{\FloatTok}[1]{\textcolor[rgb]{0.00,0.00,0.81}{#1}}
\newcommand{\FunctionTok}[1]{\textcolor[rgb]{0.00,0.00,0.00}{#1}}
\newcommand{\ImportTok}[1]{#1}
\newcommand{\InformationTok}[1]{\textcolor[rgb]{0.56,0.35,0.01}{\textbf{\textit{#1}}}}
\newcommand{\KeywordTok}[1]{\textcolor[rgb]{0.13,0.29,0.53}{\textbf{#1}}}
\newcommand{\NormalTok}[1]{#1}
\newcommand{\OperatorTok}[1]{\textcolor[rgb]{0.81,0.36,0.00}{\textbf{#1}}}
\newcommand{\OtherTok}[1]{\textcolor[rgb]{0.56,0.35,0.01}{#1}}
\newcommand{\PreprocessorTok}[1]{\textcolor[rgb]{0.56,0.35,0.01}{\textit{#1}}}
\newcommand{\RegionMarkerTok}[1]{#1}
\newcommand{\SpecialCharTok}[1]{\textcolor[rgb]{0.00,0.00,0.00}{#1}}
\newcommand{\SpecialStringTok}[1]{\textcolor[rgb]{0.31,0.60,0.02}{#1}}
\newcommand{\StringTok}[1]{\textcolor[rgb]{0.31,0.60,0.02}{#1}}
\newcommand{\VariableTok}[1]{\textcolor[rgb]{0.00,0.00,0.00}{#1}}
\newcommand{\VerbatimStringTok}[1]{\textcolor[rgb]{0.31,0.60,0.02}{#1}}
\newcommand{\WarningTok}[1]{\textcolor[rgb]{0.56,0.35,0.01}{\textbf{\textit{#1}}}}
\usepackage{graphicx,grffile}
\makeatletter
\def\maxwidth{\ifdim\Gin@nat@width>\linewidth\linewidth\else\Gin@nat@width\fi}
\def\maxheight{\ifdim\Gin@nat@height>\textheight\textheight\else\Gin@nat@height\fi}
\makeatother
% Scale images if necessary, so that they will not overflow the page
% margins by default, and it is still possible to overwrite the defaults
% using explicit options in \includegraphics[width, height, ...]{}
\setkeys{Gin}{width=\maxwidth,height=\maxheight,keepaspectratio}
% Set default figure placement to htbp
\makeatletter
\def\fps@figure{htbp}
\makeatother
\setlength{\emergencystretch}{3em} % prevent overfull lines
\providecommand{\tightlist}{%
  \setlength{\itemsep}{0pt}\setlength{\parskip}{0pt}}
\setcounter{secnumdepth}{-\maxdimen} % remove section numbering

\title{Redes Neurais Artificiais}
\usepackage{etoolbox}
\makeatletter
\providecommand{\subtitle}[1]{% add subtitle to \maketitle
  \apptocmd{\@title}{\par {\large #1 \par}}{}{}
}
\makeatother
\subtitle{Exercício 6 - ELMs com bases de dados reais}
\author{Vítor Gabriel Reis Caitité - 2016111849}
\date{1/25/2021}

\begin{document}
\maketitle

\hypertarget{funuxe7uxe3o-que-calcula-a-sauxedda-de-uma-rede-elm}{%
\subsection{Função que calcula a saída de uma rede
ELM}\label{funuxe7uxe3o-que-calcula-a-sauxedda-de-uma-rede-elm}}

Abaixo está a função que calcula a saída de uma rede ELM.

\begin{Shaded}
\begin{Highlighting}[]
\NormalTok{YELM<-}\ControlFlowTok{function}\NormalTok{(xin, Z, W, par)\{}
\NormalTok{  n<-}\KeywordTok{dim}\NormalTok{(xin)[}\DecValTok{2}\NormalTok{]}
  
  \CommentTok{# Adiciona ou não termo de polarização}
  \ControlFlowTok{if}\NormalTok{(par }\OperatorTok{==}\StringTok{ }\DecValTok{1}\NormalTok{) \{}
\NormalTok{    xin<-}\KeywordTok{cbind}\NormalTok{(}\DecValTok{1}\NormalTok{, xin) }
\NormalTok{  \}}
\NormalTok{  H<-}\KeywordTok{tanh}\NormalTok{(xin}\OperatorTok\NormalTok{Z)}
\NormalTok{  y_hat<-}\KeywordTok{sign}\NormalTok{(H }\OperatorTok\StringTok{ }\NormalTok{W)}
  \KeywordTok{return}\NormalTok{(y_hat)}
\NormalTok{\}}
\end{Highlighting}
\end{Shaded}

\hypertarget{treinamento-da-rede-elm}{%
\subsection{Treinamento da rede ELM}\label{treinamento-da-rede-elm}}

A função abaixo é uma implementação possível, em R, para o algoritmo de
treinamento de uma rede ELM. Essa função é utilizada na resolução dos
exercícios da lista.

\begin{Shaded}
\begin{Highlighting}[]
\KeywordTok{library}\NormalTok{(}\StringTok{"corpcor"}\NormalTok{)}

\NormalTok{trainELM <-}\StringTok{ }\ControlFlowTok{function}\NormalTok{(xin, yin, p, par)\{}
\NormalTok{  n <-}\StringTok{ }\KeywordTok{dim}\NormalTok{(xin) [}\DecValTok{2}\NormalTok{] }\CommentTok{# Dimensão da entrada}

  \CommentTok{#Adiciona ou não o termo de polarização}
  \ControlFlowTok{if}\NormalTok{(par }\OperatorTok{==}\StringTok{ }\DecValTok{1}\NormalTok{)\{}
\NormalTok{    xin<-}\KeywordTok{cbind}\NormalTok{(}\DecValTok{1}\NormalTok{,xin)}
\NormalTok{    Z<-}\KeywordTok{replicate}\NormalTok{(p, }\KeywordTok{runif}\NormalTok{(n}\OperatorTok{+}\DecValTok{1}\NormalTok{, }\FloatTok{-0.5}\NormalTok{, }\FloatTok{0.5}\NormalTok{))}
\NormalTok{  \}}
  \ControlFlowTok{else}\NormalTok{\{}
\NormalTok{    Z<-}\KeywordTok{replicate}\NormalTok{(p, }\KeywordTok{runif}\NormalTok{(n, }\FloatTok{-0.5}\NormalTok{, }\FloatTok{0.5}\NormalTok{))}
\NormalTok{  \}}
\NormalTok{  H<-}\KeywordTok{tanh}\NormalTok{(xin }\OperatorTok\StringTok{ }\NormalTok{Z)}
  
\NormalTok{  W<-}\KeywordTok{pseudoinverse}\NormalTok{(H)}\OperatorTok\NormalTok{yin}
  \CommentTok{#W<-(solve(t(H) %*% H) %*% t(H)) %*% yin}
  
  \KeywordTok{return}\NormalTok{(}\KeywordTok{list}\NormalTok{(W,H,Z))}
\NormalTok{\}}
\end{Highlighting}
\end{Shaded}

\hypertarget{funuxe7uxe3o-que-calcula-a-resposta-de-um-perceptron-simples}{%
\subsection{Função que calcula a resposta de um perceptron
simples}\label{funuxe7uxe3o-que-calcula-a-resposta-de-um-perceptron-simples}}

Abaixo está a função que calcula a resposta de um perceptron simples e é
utilizada nas questões da lista.

\begin{Shaded}
\begin{Highlighting}[]
\CommentTok{#Função que calcula a resposta de um perceptron simples.}
\NormalTok{yperceptron <-}\StringTok{ }\ControlFlowTok{function}\NormalTok{(xvec, w, par)\{}
  \CommentTok{# xvec: vetor de entrada}
  \CommentTok{# w: vetor de pesos}
  \CommentTok{# par: se adiciona ou não o vetor de 1s na entrada }
  \CommentTok{# yperceptron: resposta do perceptron}
  \ControlFlowTok{if}\NormalTok{ ( par}\OperatorTok{==}\DecValTok{1}\NormalTok{)\{}
\NormalTok{    xvec<-}\KeywordTok{cbind}\NormalTok{ ( }\DecValTok{1}\NormalTok{ , xvec )}
\NormalTok{  \} }
\NormalTok{  u <-}\StringTok{ }\NormalTok{xvec }\OperatorTok\StringTok{ }\NormalTok{w}
\NormalTok{  y <-}\StringTok{ }\FloatTok{1.0} \OperatorTok{*}\StringTok{ }\NormalTok{(u}\OperatorTok{>=}\DecValTok{0}\NormalTok{)}
  \KeywordTok{return}\NormalTok{(}\KeywordTok{as.matrix}\NormalTok{(y))}
\NormalTok{\}}
\end{Highlighting}
\end{Shaded}

\hypertarget{treinamento-do-perceptron}{%
\subsection{Treinamento do perceptron}\label{treinamento-do-perceptron}}

A função abaixo é uma implementação possível, em R, para o algoritmo de
treinamento do Perceptron. Essa função é utilizada na resolução dos
exercícios da lista.

\begin{Shaded}
\begin{Highlighting}[]
\NormalTok{trainPerceptron <-}\StringTok{ }\ControlFlowTok{function}\NormalTok{ ( xin , yd , eta , tol , maxepocas , par )}
\NormalTok{\{}
\NormalTok{  dimxin<-}\KeywordTok{dim}\NormalTok{( xin )}
\NormalTok{  N <-dimxin[ }\DecValTok{1}\NormalTok{ ]}
\NormalTok{  n<-dimxin[ }\DecValTok{2}\NormalTok{ ]}
  \ControlFlowTok{if}\NormalTok{ ( par}\OperatorTok{==}\DecValTok{1}\NormalTok{)\{}
\NormalTok{    wt<-}\KeywordTok{as.matrix}\NormalTok{ ( }\KeywordTok{runif}\NormalTok{(n}\OperatorTok{+}\DecValTok{1}\NormalTok{) }\OperatorTok{-}\StringTok{ }\FloatTok{0.5}\NormalTok{)}
\NormalTok{    xin<-}\KeywordTok{cbind}\NormalTok{ ( }\DecValTok{1}\NormalTok{ , xin )}
\NormalTok{  \} }\ControlFlowTok{else}\NormalTok{ \{}
\NormalTok{    wt<-}\KeywordTok{as.matrix}\NormalTok{ ( }\KeywordTok{runif}\NormalTok{ ( n ) }\OperatorTok{-}\StringTok{ }\FloatTok{0.5}\NormalTok{) }
\NormalTok{  \}}
\NormalTok{  nepocas<-}\DecValTok{0}
\NormalTok{  eepoca<-tol }\OperatorTok{+}\StringTok{ }\DecValTok{1}
  
\NormalTok{  evec<-}\KeywordTok{matrix}\NormalTok{ ( }\DataTypeTok{nrow =}\DecValTok{1}\NormalTok{ , }\DataTypeTok{ncol=}\NormalTok{maxepocas )}
  \ControlFlowTok{while}\NormalTok{( ( nepocas }\OperatorTok{<}\StringTok{ }\NormalTok{maxepocas ) }\OperatorTok{&&}\StringTok{ }\NormalTok{( eepoca}\OperatorTok{>}\NormalTok{tol ) )}
\NormalTok{  \{}
\NormalTok{    ei2<-}\DecValTok{0}
\NormalTok{    xseq<-}\KeywordTok{sample}\NormalTok{(N)}
    \ControlFlowTok{for}\NormalTok{ ( i }\ControlFlowTok{in} \DecValTok{1}\OperatorTok{:}\NormalTok{N)}
\NormalTok{    \{}
\NormalTok{      irand<-xseq [i]}
\NormalTok{      yhati<-}\FloatTok{1.0} \OperatorTok{*}\StringTok{ }\NormalTok{( ( xin[}
\NormalTok{        irand , ] }\OperatorTok\StringTok{ }\NormalTok{wt ) }\OperatorTok{>=}\StringTok{ }\DecValTok{0}\NormalTok{ )}
\NormalTok{      ei<-yd[irand]}\OperatorTok{-}\StringTok{ }\NormalTok{yhati}
\NormalTok{      dw<-}\KeywordTok{as.vector}\NormalTok{(eta) }\OperatorTok{*}\StringTok{ }\KeywordTok{as.vector}\NormalTok{(ei) }\OperatorTok{*}\StringTok{ }\NormalTok{xin[ irand  , ]}
\NormalTok{      wt<-wt}\OperatorTok{+}\NormalTok{dw}
\NormalTok{      ei2<-ei2 }\OperatorTok{+}\StringTok{ }\NormalTok{ei }\OperatorTok{*}\StringTok{ }\NormalTok{ei}
\NormalTok{    \}}
\NormalTok{    nepocas<-nepocas}\OperatorTok{+}\DecValTok{1}
\NormalTok{    evec[ nepocas ]<-ei2}\OperatorTok{/}\NormalTok{N}
    
\NormalTok{    eepoca<-evec[nepocas]}
\NormalTok{  \}}
\NormalTok{  retlist<-}\KeywordTok{list}\NormalTok{ ( wt, evec[ }\DecValTok{1}\OperatorTok{:}\NormalTok{nepocas ] )}
  \KeywordTok{return}\NormalTok{ (retlist)}
\NormalTok{\}}
\end{Highlighting}
\end{Shaded}

\hypertarget{enunciado-exercuxedcio-6}{%
\subsection{Enunciado Exercício 6}\label{enunciado-exercuxedcio-6}}

O objetivo dos exercícios desta semana é utilizar as ELMs para resolver
problemas multidimensionais, a partir de bases de dados reais. As bases
de dados devem ser baixadas do repositório UCI Machine Learning
Repository (\url{https://archive.ics.uci.edu/ml/datasets.php}). A
primeira base de dados a ser estudada é a base Breast Cancer
(diagnostic), disponível no link:

\url{https://archive.ics.uci.edu/ml/datasets/Breast+Cancer+Wisconsin+\%28Diagnostic\%29}

Para esta base, os alunos deverão dividir de forma aleatória os dados
entre treinamento e teste e comparar as acurácias de treinamento e teste
para diferentes valores do hiperparâmetro que controla o número de
neurônios. Os valores de acurácia devem ser apresentados na forma de
media ± desvio\_padrao para, pelo menos, cinco execuções diferentes.

Algumas perguntas que devem ser respondidas são:

• Com quantos neurônios (aproximadamente) a acurácia de teste aparenta
ser máxima?

• O que acontece com os valores de acurácia de treinamento e teste
conforme aumentamos progressivamente o número de neurônios (por exemplo,
para 5, 10, 30, 50, 100, 300 neurônios)?

O mesmo deve ser feito para a base Statlog (Heart), disponível no link:

\url{https://archive.ics.uci.edu/ml/datasets/Statlog+\%28Heart\%29}

Além das Extreme Learning Machines, os alunos deverão, também, treinar,
utilizando a rotina desenvolvida para as atividades anteriores, um
perceptron, e avaliar seu desempenho na solução dos dois problemas,
comparado às ELMs.

Por questões de convergência, pode ser necessário escalonar os valores
dos atributos para que fiquem restritos entre 0 e 1. Para tanto, uma
possibilidade é utilizar a forma abaixo:

z\_i = {[}x\_i − min(x){]}/{[}max(x) − min(x){]}

\hypertarget{elm-com-base-de-dados-breast-cancer-diagnostic}{%
\subsection{ELM com base de dados Breast Cancer
(diagnostic)}\label{elm-com-base-de-dados-breast-cancer-diagnostic}}

Ulizando a base de dados Breast Cancer (diagnostic) foi desenvolvida uma
ELM para classificar um tumor em maligno ou benigno. Essa base de dados
é composta de 32 atributos, sendo eles: ID, diagnóstico (``M''-maligno,
ou ``B''-benigno) e 30 características de entrada com valores reais.
Foram utilizados 70\% dos dados para treino e 30\% para teste. Além
disso, variou-se progressivamente o número de neurônios da seguinte
forma, 5, 10, 30, 50, 100, 150, 200 e 300 neurônios. Para cada número de
neurônios foram realizados 20 execuções diferentes de treinamento e
teste, e os valores de acurácia, para cada caso, foram apresentados na
forma de média ± desvio\_padrão. Os resultados e o script desenvolvido
podem ser vistos abaixo.

\begin{Shaded}
\begin{Highlighting}[]
\KeywordTok{rm}\NormalTok{(}\DataTypeTok{list=}\KeywordTok{ls}\NormalTok{())}
\KeywordTok{source}\NormalTok{(}\StringTok{"~/Documents/UFMG/9/Redes Neurais/exemplos/trainELM.R"}\NormalTok{)}
\KeywordTok{source}\NormalTok{(}\StringTok{"~/Documents/UFMG/9/Redes Neurais/exemplos/YELM.R"}\NormalTok{)}
\KeywordTok{library}\NormalTok{(caret)}
\end{Highlighting}
\end{Shaded}

\begin{verbatim}
## Loading required package: lattice
\end{verbatim}

\begin{verbatim}
## Loading required package: ggplot2
\end{verbatim}

\begin{Shaded}
\begin{Highlighting}[]
\CommentTok{# Carregando base de dados:}
\NormalTok{path <-}\StringTok{ }\KeywordTok{file.path}\NormalTok{(}\StringTok{"~/Documents/UFMG/9/Redes Neurais/listas/lista 6/cancer"}\NormalTok{, }\StringTok{"wdbc.csv"}\NormalTok{)}
\NormalTok{data <-}\StringTok{ }\KeywordTok{read.csv}\NormalTok{(path)}

\CommentTok{# Separando dados de entrada e saída:}
\NormalTok{x_all <-}\StringTok{ }\KeywordTok{as.matrix}\NormalTok{(data[}\DecValTok{1}\OperatorTok{:}\DecValTok{569}\NormalTok{, }\DecValTok{3}\OperatorTok{:}\DecValTok{32}\NormalTok{])}
\NormalTok{class <-}\StringTok{ }\KeywordTok{as.matrix}\NormalTok{(data[}\DecValTok{1}\OperatorTok{:}\DecValTok{569}\NormalTok{, }\DecValTok{2}\NormalTok{])}
\NormalTok{y_all <-}\StringTok{ }\KeywordTok{rep}\NormalTok{(}\DecValTok{0}\NormalTok{,}\DecValTok{569}\NormalTok{)}
\ControlFlowTok{for}\NormalTok{ (count }\ControlFlowTok{in} \DecValTok{1}\OperatorTok{:}\KeywordTok{length}\NormalTok{(class)) \{}
  \ControlFlowTok{if}\NormalTok{ (class[count] }\OperatorTok{==}\StringTok{ 'M'}\NormalTok{ )\{}
\NormalTok{    y_all[count] =}\StringTok{ }\DecValTok{-1}
\NormalTok{  \}}
  \ControlFlowTok{else} \ControlFlowTok{if}\NormalTok{(class[count] }\OperatorTok{==}\StringTok{ 'B'}\NormalTok{)\{}
\NormalTok{    y_all[count] =}\StringTok{ }\DecValTok{1}
\NormalTok{  \}}
\NormalTok{\}}

\ControlFlowTok{for}\NormalTok{ (p }\ControlFlowTok{in} \KeywordTok{c}\NormalTok{(}\DecValTok{5}\NormalTok{,}\DecValTok{10}\NormalTok{,}\DecValTok{30}\NormalTok{,}\DecValTok{50}\NormalTok{,}\DecValTok{100}\NormalTok{,}\DecValTok{150}\NormalTok{,}\DecValTok{200}\NormalTok{,}\DecValTok{300}\NormalTok{))\{}
  \CommentTok{# Realiza pelo 20 execuções diferentes}
\NormalTok{  accuracy_train <-}\StringTok{ }\KeywordTok{rep}\NormalTok{(}\DecValTok{0}\NormalTok{, }\DecValTok{20}\NormalTok{)}
\NormalTok{  accuracy_test <-}\StringTok{ }\KeywordTok{rep}\NormalTok{(}\DecValTok{0}\NormalTok{, }\DecValTok{20}\NormalTok{)}
  \ControlFlowTok{for}\NormalTok{(execution }\ControlFlowTok{in} \DecValTok{1}\OperatorTok{:}\DecValTok{20}\NormalTok{)\{}
    \CommentTok{# Separando dados entre treino e teste aleatoriamente:}
\NormalTok{    positions_train <-}\StringTok{ }\KeywordTok{createDataPartition}\NormalTok{(}\DecValTok{1}\OperatorTok{:}\DecValTok{569}\NormalTok{,}\DataTypeTok{p=}\NormalTok{.}\DecValTok{7}\NormalTok{)}
\NormalTok{    length_train <-}\StringTok{ }\KeywordTok{length}\NormalTok{(positions_train}\OperatorTok{$}\NormalTok{Resample1)}
\NormalTok{    length_test <-}\StringTok{ }\KeywordTok{length}\NormalTok{(y_all) }\OperatorTok{-}\StringTok{ }\NormalTok{length_train}
\NormalTok{    x_train <-}\StringTok{ }\KeywordTok{matrix}\NormalTok{(}\KeywordTok{rep}\NormalTok{(}\DecValTok{0}\NormalTok{, }\DecValTok{30}\OperatorTok{*}\NormalTok{length_train), }\DataTypeTok{ncol=}\DecValTok{30}\NormalTok{, }\DataTypeTok{nrow=}\NormalTok{length_train)}
\NormalTok{    y_train <-}\StringTok{ }\KeywordTok{rep}\NormalTok{(}\DecValTok{0}\NormalTok{, length_train)}
\NormalTok{    x_test <-}\StringTok{ }\KeywordTok{matrix}\NormalTok{(}\KeywordTok{rep}\NormalTok{(}\DecValTok{0}\NormalTok{, (}\DecValTok{30}\OperatorTok{*}\NormalTok{length_test)), }\DataTypeTok{ncol=}\DecValTok{30}\NormalTok{, }\DataTypeTok{nrow=}\NormalTok{(}\KeywordTok{length}\NormalTok{(y_all) }\OperatorTok{-}\StringTok{ }\NormalTok{length_train))}
\NormalTok{    y_test <-}\StringTok{ }\KeywordTok{rep}\NormalTok{(}\DecValTok{0}\NormalTok{, (}\KeywordTok{length}\NormalTok{(y_all) }\OperatorTok{-}\StringTok{ }\NormalTok{length_train))}
\NormalTok{    index_train <-}\StringTok{ }\DecValTok{1}
\NormalTok{    index_test <-}\StringTok{ }\DecValTok{1}
    \ControlFlowTok{for}\NormalTok{ (count }\ControlFlowTok{in} \DecValTok{1}\OperatorTok{:}\KeywordTok{length}\NormalTok{(y_all)) \{}
      \ControlFlowTok{if}\NormalTok{ (index_train }\OperatorTok{<=}\StringTok{ }\NormalTok{length_train }\OperatorTok{&&}\StringTok{ }\NormalTok{count }\OperatorTok{==}\StringTok{ }\NormalTok{positions_train}\OperatorTok{$}\NormalTok{Resample1[index_train])\{}
\NormalTok{        x_train[index_train, ] <-}\StringTok{ }\NormalTok{x_all[count, }\DecValTok{1}\OperatorTok{:}\DecValTok{30}\NormalTok{ ]}
\NormalTok{        y_train[index_train] <-}\StringTok{ }\NormalTok{y_all[count]}
\NormalTok{        index_train =}\StringTok{ }\NormalTok{index_train }\OperatorTok{+}\StringTok{ }\DecValTok{1}
\NormalTok{      \} }\ControlFlowTok{else}\NormalTok{ \{}
\NormalTok{        x_test[index_test, ] <-}\StringTok{ }\NormalTok{x_all[count, }\DecValTok{1}\OperatorTok{:}\DecValTok{30}\NormalTok{ ]}
\NormalTok{        y_test[index_test] <-}\StringTok{ }\NormalTok{y_all[count]}
\NormalTok{        index_test =}\StringTok{ }\NormalTok{index_test }\OperatorTok{+}\StringTok{ }\DecValTok{1}    
\NormalTok{      \}}
\NormalTok{    \}}
    
    \CommentTok{# Treinando modelo:}
\NormalTok{    retlist<-}\KeywordTok{trainELM}\NormalTok{(x_train, y_train, p, }\DecValTok{1}\NormalTok{)}
\NormalTok{    W<-retlist[[}\DecValTok{1}\NormalTok{]]}
\NormalTok{    H<-retlist[[}\DecValTok{2}\NormalTok{]]}
\NormalTok{    Z<-retlist[[}\DecValTok{3}\NormalTok{]]}
    
    \CommentTok{# Calculando acurácia de treinamento}
\NormalTok{    y_hat_train <-}\StringTok{ }\KeywordTok{as.matrix}\NormalTok{(}\KeywordTok{YELM}\NormalTok{(x_train, Z, W, }\DecValTok{1}\NormalTok{), }\DataTypeTok{nrow =}\NormalTok{ length_train, }\DataTypeTok{ncol =} \DecValTok{1}\NormalTok{)}
\NormalTok{    accuracy_train[execution]<-((}\KeywordTok{sum}\NormalTok{(}\KeywordTok{abs}\NormalTok{(y_hat_train }\OperatorTok{+}\StringTok{ }\NormalTok{y_train)))}\OperatorTok{/}\DecValTok{2}\NormalTok{)}\OperatorTok{/}\NormalTok{length_train}
    \CommentTok{#print(paste("Acurácia de treinamento para execução", execution, "com", p, "nerônios:", accuracy_train))}
    
    \CommentTok{# Calculando acurácia de Teste:}
\NormalTok{    y_hat_test <-}\StringTok{ }\KeywordTok{as.matrix}\NormalTok{(}\KeywordTok{YELM}\NormalTok{(x_test, Z, W, }\DecValTok{1}\NormalTok{), }\DataTypeTok{nrow =}\NormalTok{ length_test, }\DataTypeTok{ncol =} \DecValTok{1}\NormalTok{)}
\NormalTok{    accuracy_test[execution]<-((}\KeywordTok{sum}\NormalTok{(}\KeywordTok{abs}\NormalTok{(y_hat_test }\OperatorTok{+}\StringTok{ }\NormalTok{y_test)))}\OperatorTok{/}\DecValTok{2}\NormalTok{)}\OperatorTok{/}\NormalTok{length_test}
    \CommentTok{#print(paste("Acurácia de teste para execução", execution, "com", p, "nerônios:", accuracy_test))}
\NormalTok{  \}}
  \CommentTok{# Média das acurácias}
\NormalTok{  mean_accuracy_train <-}\StringTok{ }\KeywordTok{mean}\NormalTok{(accuracy_train) }\OperatorTok{*}\StringTok{ }\DecValTok{100}
\NormalTok{  mean_accuracy_test <-}\StringTok{ }\KeywordTok{mean}\NormalTok{(accuracy_test) }\OperatorTok{*}\StringTok{ }\DecValTok{100}
  
  \CommentTok{# Desvio Padrão das acurácias}
\NormalTok{  sd_accuracy_train <-}\StringTok{ }\KeywordTok{sd}\NormalTok{(accuracy_train) }\OperatorTok{*}\StringTok{ }\DecValTok{100}
\NormalTok{  sd_accuracy_test <-}\StringTok{ }\KeywordTok{sd}\NormalTok{(accuracy_test) }\OperatorTok{*}\StringTok{ }\DecValTok{100}
  
  \KeywordTok{print}\NormalTok{(}\KeywordTok{paste}\NormalTok{(}\StringTok{"Acurácia de treinamento do modelo com"}\NormalTok{, p, }\StringTok{"neurônios:"}\NormalTok{, mean_accuracy_train, }\StringTok{"%"}\NormalTok{, }\StringTok{"±"}\NormalTok{, sd_accuracy_train, }\StringTok{"%"}\NormalTok{))}
  \KeywordTok{print}\NormalTok{(}\KeywordTok{paste}\NormalTok{(}\StringTok{"Acurácia de teste do modelo com"}\NormalTok{, p, }\StringTok{"neurônios:"}\NormalTok{, mean_accuracy_test, }\StringTok{"%"}\NormalTok{, }\StringTok{"±"}\NormalTok{, sd_accuracy_test, }\StringTok{"%"}\NormalTok{))}
\NormalTok{\}}
\end{Highlighting}
\end{Shaded}

\begin{verbatim}
## [1] "Acurácia de treinamento do modelo com 5 neurônios: 62.3192019950125 % ± 2.19185726175669 %"
## [1] "Acurácia de teste do modelo com 5 neurônios: 60.1488095238095 % ± 4.48553838471902 %"
## [1] "Acurácia de treinamento do modelo com 10 neurônios: 65.4613466334165 % ± 2.87025468618688 %"
## [1] "Acurácia de teste do modelo com 10 neurônios: 62.6190476190476 % ± 3.57977304208244 %"
## [1] "Acurácia de treinamento do modelo com 30 neurônios: 73.0299251870324 % ± 3.3703730986751 %"
## [1] "Acurácia de teste do modelo com 30 neurônios: 69.4642857142857 % ± 5.31029361559272 %"
## [1] "Acurácia de treinamento do modelo com 50 neurônios: 78.1546134663342 % ± 2.7250654257246 %"
## [1] "Acurácia de teste do modelo com 50 neurônios: 73.2738095238095 % ± 3.64839783725019 %"
## [1] "Acurácia de treinamento do modelo com 100 neurônios: 83.4413965087282 % ± 2.45727335723328 %"
## [1] "Acurácia de teste do modelo com 100 neurônios: 74.4642857142857 % ± 4.92703958372101 %"
## [1] "Acurácia de treinamento do modelo com 150 neurônios: 86.3092269326683 % ± 2.66740290068355 %"
## [1] "Acurácia de teste do modelo com 150 neurônios: 72.6190476190476 % ± 3.80403740267182 %"
## [1] "Acurácia de treinamento do modelo com 200 neurônios: 89.7381546134663 % ± 1.71112263767837 %"
## [1] "Acurácia de teste do modelo com 200 neurônios: 72.8571428571428 % ± 4.7712940015741 %"
## [1] "Acurácia de treinamento do modelo com 300 neurônios: 93.8403990024938 % ± 1.68184495920265 %"
## [1] "Acurácia de teste do modelo com 300 neurônios: 69.9107142857143 % ± 3.15251492306318 %"
\end{verbatim}

\hypertarget{discussuxe3o}{%
\paragraph{\texorpdfstring{\textbf{\emph{Discussão:}}}{Discussão:}}\label{discussuxe3o}}

Pôde-se perceber que a acurácia de treinamento também vai aumentando
progressivamente a medida que aumenta-se o número de neurônios. Contudo,
o mesmo não ocorre para a acurácia de teste. Uma explicação plausível
para isso é que um modelo com um número muito elevado de neurônios pode
se tornar super ajustado aos dados de treinamento e não funcionar da
maneira esperada com dados de teste. Aparentemente, a acurácia média
máxima de teste foi obtida para o número de 100 neurônios.

\hypertarget{perceptron-simples-com-base-de-dados-breast-cancer-diagnostic}{%
\subsection{Perceptron simples com base de dados Breast Cancer
(diagnostic)}\label{perceptron-simples-com-base-de-dados-breast-cancer-diagnostic}}

Nessa etapa foi treinado um perceptron simples, utilizando a rotina de
treinamento de perceptron mostrada no início desse documento e a base de
dados da questão anterior, para também classificar um tumor em maligno
ou benigno. Foram utilizados 70\% dos dados para treino e 30\% para
teste. Além disso, foram realizados 20 execuções diferentes de
treinamento e teste, e os valores de acurácia (de treinamento e teste),
foram apresentados na forma de média ± desvio\_padrão. Os resultados e o
script desenvolvido podem ser vistos abaixo.

\begin{Shaded}
\begin{Highlighting}[]
\KeywordTok{rm}\NormalTok{(}\DataTypeTok{list=}\KeywordTok{ls}\NormalTok{())}
\KeywordTok{source}\NormalTok{(}\StringTok{"~/Documents/UFMG/9/Redes Neurais/listas/lista 4/trainPerceptron.R"}\NormalTok{)}
\KeywordTok{source}\NormalTok{(}\StringTok{"~/Documents/UFMG/9/Redes Neurais/listas/lista 4/yperceptron.R"}\NormalTok{)}
\KeywordTok{library}\NormalTok{(caret)}

\CommentTok{# Carregando base de dados:}
\NormalTok{path <-}\StringTok{ }\KeywordTok{file.path}\NormalTok{(}\StringTok{"~/Documents/UFMG/9/Redes Neurais/listas/lista 6/cancer"}\NormalTok{, }\StringTok{"wdbc.csv"}\NormalTok{)}
\NormalTok{data <-}\StringTok{ }\KeywordTok{read.csv}\NormalTok{(path)}

\CommentTok{# Separando dados de entrada e saída:}
\NormalTok{x_all <-}\StringTok{ }\KeywordTok{as.matrix}\NormalTok{(data[}\DecValTok{1}\OperatorTok{:}\DecValTok{569}\NormalTok{, }\DecValTok{3}\OperatorTok{:}\DecValTok{32}\NormalTok{])}
\NormalTok{class <-}\StringTok{ }\KeywordTok{as.matrix}\NormalTok{(data[}\DecValTok{1}\OperatorTok{:}\DecValTok{569}\NormalTok{, }\DecValTok{2}\NormalTok{])}
\NormalTok{y_all <-}\StringTok{ }\KeywordTok{rep}\NormalTok{(}\DecValTok{0}\NormalTok{,}\DecValTok{569}\NormalTok{)}
\ControlFlowTok{for}\NormalTok{ (count }\ControlFlowTok{in} \DecValTok{1}\OperatorTok{:}\KeywordTok{length}\NormalTok{(class)) \{}
  \ControlFlowTok{if}\NormalTok{ (class[count] }\OperatorTok{==}\StringTok{ 'M'}\NormalTok{ )\{}
\NormalTok{    y_all[count] =}\StringTok{ }\DecValTok{0}
\NormalTok{  \}}
  \ControlFlowTok{else} \ControlFlowTok{if}\NormalTok{(class[count] }\OperatorTok{==}\StringTok{ 'B'}\NormalTok{)\{}
\NormalTok{    y_all[count] =}\StringTok{ }\DecValTok{1}
\NormalTok{  \}}
\NormalTok{\}}

\CommentTok{# Realiza pelo 20 execuções diferentes}
\NormalTok{accuracy_train <-}\StringTok{ }\KeywordTok{rep}\NormalTok{(}\DecValTok{0}\NormalTok{, }\DecValTok{20}\NormalTok{)}
\NormalTok{accuracy_test <-}\StringTok{ }\KeywordTok{rep}\NormalTok{(}\DecValTok{0}\NormalTok{, }\DecValTok{20}\NormalTok{)}
\ControlFlowTok{for}\NormalTok{(execution }\ControlFlowTok{in} \DecValTok{1}\OperatorTok{:}\DecValTok{20}\NormalTok{)\{}
  \CommentTok{# Separando dados entre treino e teste aleatoriamente:}
\NormalTok{  positions_train <-}\StringTok{ }\KeywordTok{createDataPartition}\NormalTok{(}\DecValTok{1}\OperatorTok{:}\DecValTok{569}\NormalTok{,}\DataTypeTok{p=}\NormalTok{.}\DecValTok{7}\NormalTok{)}
\NormalTok{  length_train <-}\StringTok{ }\KeywordTok{length}\NormalTok{(positions_train}\OperatorTok{$}\NormalTok{Resample1)}
\NormalTok{  length_test <-}\StringTok{ }\KeywordTok{length}\NormalTok{(y_all) }\OperatorTok{-}\StringTok{ }\NormalTok{length_train}
\NormalTok{  x_train <-}\StringTok{ }\KeywordTok{matrix}\NormalTok{(}\KeywordTok{rep}\NormalTok{(}\DecValTok{0}\NormalTok{, }\DecValTok{30}\OperatorTok{*}\NormalTok{length_train), }\DataTypeTok{ncol=}\DecValTok{30}\NormalTok{, }\DataTypeTok{nrow=}\NormalTok{length_train)}
\NormalTok{  y_train <-}\StringTok{ }\KeywordTok{rep}\NormalTok{(}\DecValTok{0}\NormalTok{, length_train)}
\NormalTok{  x_test <-}\StringTok{ }\KeywordTok{matrix}\NormalTok{(}\KeywordTok{rep}\NormalTok{(}\DecValTok{0}\NormalTok{, (}\DecValTok{30}\OperatorTok{*}\NormalTok{length_test)), }\DataTypeTok{ncol=}\DecValTok{30}\NormalTok{, }\DataTypeTok{nrow=}\NormalTok{(}\KeywordTok{length}\NormalTok{(y_all) }\OperatorTok{-}\StringTok{ }\NormalTok{length_train))}
\NormalTok{  y_test <-}\StringTok{ }\KeywordTok{rep}\NormalTok{(}\DecValTok{0}\NormalTok{, (}\KeywordTok{length}\NormalTok{(y_all) }\OperatorTok{-}\StringTok{ }\NormalTok{length_train))}
\NormalTok{  index_train <-}\StringTok{ }\DecValTok{1}
\NormalTok{  index_test <-}\StringTok{ }\DecValTok{1}
  \ControlFlowTok{for}\NormalTok{ (count }\ControlFlowTok{in} \DecValTok{1}\OperatorTok{:}\KeywordTok{length}\NormalTok{(y_all)) \{}
    \ControlFlowTok{if}\NormalTok{ (index_train }\OperatorTok{<=}\StringTok{ }\NormalTok{length_train }\OperatorTok{&&}\StringTok{ }\NormalTok{count }\OperatorTok{==}\StringTok{ }\NormalTok{positions_train}\OperatorTok{$}\NormalTok{Resample1[index_train])\{}
\NormalTok{      x_train[index_train, ] <-}\StringTok{ }\NormalTok{x_all[count, }\DecValTok{1}\OperatorTok{:}\DecValTok{30}\NormalTok{ ]}
\NormalTok{      y_train[index_train] <-}\StringTok{ }\NormalTok{y_all[count]}
\NormalTok{      index_train =}\StringTok{ }\NormalTok{index_train }\OperatorTok{+}\StringTok{ }\DecValTok{1}
\NormalTok{    \} }\ControlFlowTok{else}\NormalTok{ \{}
\NormalTok{      x_test[index_test, ] <-}\StringTok{ }\NormalTok{x_all[count, }\DecValTok{1}\OperatorTok{:}\DecValTok{30}\NormalTok{ ]}
\NormalTok{      y_test[index_test] <-}\StringTok{ }\NormalTok{y_all[count]}
\NormalTok{      index_test =}\StringTok{ }\NormalTok{index_test }\OperatorTok{+}\StringTok{ }\DecValTok{1}    
\NormalTok{    \}}
\NormalTok{  \}}
    
  \CommentTok{# Treinando modelo:}
\NormalTok{  retlist<-}\KeywordTok{trainPerceptron}\NormalTok{(x_train, y_train, }\FloatTok{0.1}\NormalTok{, }\FloatTok{0.01}\NormalTok{, }\DecValTok{1000}\NormalTok{, }\DecValTok{1}\NormalTok{)}
\NormalTok{  W<-retlist[[}\DecValTok{1}\NormalTok{]]}
    
  \CommentTok{# Calculando acurácia de treinamento}
\NormalTok{  y_hat_train <-}\StringTok{ }\KeywordTok{as.matrix}\NormalTok{(}\KeywordTok{yperceptron}\NormalTok{(x_train, W, }\DecValTok{1}\NormalTok{), }\DataTypeTok{nrow =}\NormalTok{ length_train, }\DataTypeTok{ncol =} \DecValTok{1}\NormalTok{)}
\NormalTok{  accuracy_train[execution]<-}\DecValTok{1}\OperatorTok{-}\NormalTok{((}\KeywordTok{t}\NormalTok{(y_hat_train}\OperatorTok{-}\NormalTok{y_train) }\OperatorTok\StringTok{ }\NormalTok{(y_hat_train}\OperatorTok{-}\NormalTok{y_train))}\OperatorTok{/}\NormalTok{length_train)}
  \CommentTok{#print(paste("Acurácia de treinamento para execução", execution, "com", p, "nerônios:", accuracy_train))}
    
  \CommentTok{# Calculando acurácia de Teste:}
\NormalTok{  y_hat_test <-}\StringTok{ }\KeywordTok{as.matrix}\NormalTok{(}\KeywordTok{yperceptron}\NormalTok{(x_test, W, }\DecValTok{1}\NormalTok{), }\DataTypeTok{nrow =}\NormalTok{ length_test, }\DataTypeTok{ncol =} \DecValTok{1}\NormalTok{)}
\NormalTok{  accuracy_test[execution]<-}\DecValTok{1}\OperatorTok{-}\NormalTok{((}\KeywordTok{t}\NormalTok{(y_hat_test}\OperatorTok{-}\NormalTok{y_test) }\OperatorTok\StringTok{ }\NormalTok{(y_hat_test}\OperatorTok{-}\NormalTok{y_test))}\OperatorTok{/}\NormalTok{length_test)}
  \CommentTok{#print(paste("Acurácia de teste para execução", execution, "com", p, "nerônios:", accuracy_test))}
\NormalTok{\}}
\CommentTok{# Média das acurácias}
\NormalTok{mean_accuracy_train <-}\StringTok{ }\KeywordTok{mean}\NormalTok{(accuracy_train) }\OperatorTok{*}\StringTok{ }\DecValTok{100}
\NormalTok{mean_accuracy_test <-}\StringTok{ }\KeywordTok{mean}\NormalTok{(accuracy_test) }\OperatorTok{*}\StringTok{ }\DecValTok{100}
  
\CommentTok{# Desvio Padrão das acurácias}
\NormalTok{sd_accuracy_train <-}\StringTok{ }\KeywordTok{sd}\NormalTok{(accuracy_train) }\OperatorTok{*}\StringTok{ }\DecValTok{100}
\NormalTok{sd_accuracy_test <-}\StringTok{ }\KeywordTok{sd}\NormalTok{(accuracy_test) }\OperatorTok{*}\StringTok{ }\DecValTok{100}
  
\KeywordTok{print}\NormalTok{(}\KeywordTok{paste}\NormalTok{(}\StringTok{"Acurácia de treinamento do modelo com perceptron simples"}\NormalTok{, mean_accuracy_train, }\StringTok{"%"}\NormalTok{, }\StringTok{"±"}\NormalTok{, sd_accuracy_train, }\StringTok{"%"}\NormalTok{))}
\end{Highlighting}
\end{Shaded}

\begin{verbatim}
## [1] "Acurácia de treinamento do modelo com perceptron simples 89.0149625935162 % ± 7.42132633443557 %"
\end{verbatim}

\begin{Shaded}
\begin{Highlighting}[]
\KeywordTok{print}\NormalTok{(}\KeywordTok{paste}\NormalTok{(}\StringTok{"Acurácia de teste do modelo com perceptron simples"}\NormalTok{, mean_accuracy_test, }\StringTok{"%"}\NormalTok{, }\StringTok{"±"}\NormalTok{, sd_accuracy_test, }\StringTok{"%"}\NormalTok{))}
\end{Highlighting}
\end{Shaded}

\begin{verbatim}
## [1] "Acurácia de teste do modelo com perceptron simples 87.0535714285714 % ± 4.72603580487172 %"
\end{verbatim}

\hypertarget{discussuxe3o-1}{%
\paragraph{\texorpdfstring{\textbf{\emph{Discussão:}}}{Discussão:}}\label{discussuxe3o-1}}

Pôde-se perceber que obteve-se uma acurácia de treinamento de cerca de
90\%, assim como para a ELM com 150 neurônios. Contudo, obteve-se uma
acurácia de teste também próxima dos 90\%, o que é significativamente
melhor do que a obtida para as ELMs.

\hypertarget{elm-com-base-de-dados-statlog-heart}{%
\subsection{ELM com base de dados Statlog
(Heart)}\label{elm-com-base-de-dados-statlog-heart}}

Ulizando a base de dados Statlog (Heart), foi desenvolvida uma ELM para
indicar a presença ou ausência de problemas de coração, com base em uma
série de caracteísticas. Essa base de dados é composta de 13
caracteristicas (features) e a variável de predição (que assume os
valores: 1 - ausência ou 2 - presença de doença no coração). Foram
utilizados 70\% dos dados para treino e 30\% para teste. Além disso,
variou-se progressivamente o número de neurônios da seguinte forma, 5,
10, 30, 50, 100, 150, 200 e 300 neurônios. Para cada número de neurônios
foram realizados 20 execuções diferentes de treinamento e teste, e os
valores de acurácia, para cada caso, foram apresentados na forma de
média ± desvio\_padrão. Os resultados e o script desenvolvido podem ser
vistos abaixo.

\begin{Shaded}
\begin{Highlighting}[]
\KeywordTok{rm}\NormalTok{(}\DataTypeTok{list=}\KeywordTok{ls}\NormalTok{())}
\KeywordTok{source}\NormalTok{(}\StringTok{"~/Documents/UFMG/9/Redes Neurais/exemplos/trainELM.R"}\NormalTok{)}
\KeywordTok{source}\NormalTok{(}\StringTok{"~/Documents/UFMG/9/Redes Neurais/exemplos/YELM.R"}\NormalTok{)}
\KeywordTok{library}\NormalTok{(caret)}

\CommentTok{# Carregando base de dados:}
\NormalTok{path <-}\StringTok{ }\KeywordTok{file.path}\NormalTok{(}\StringTok{"~/Documents/UFMG/9/Redes Neurais/listas/lista 6/heart"}\NormalTok{, }\StringTok{"heart.csv"}\NormalTok{)}
\NormalTok{data <-}\StringTok{ }\KeywordTok{read.csv}\NormalTok{(path)}

\CommentTok{# Separando dados de entrada e saída:}
\NormalTok{x_all <-}\StringTok{ }\KeywordTok{as.matrix}\NormalTok{(data[}\DecValTok{1}\OperatorTok{:}\DecValTok{270}\NormalTok{, }\DecValTok{1}\OperatorTok{:}\DecValTok{13}\NormalTok{])}
\NormalTok{class <-}\StringTok{ }\KeywordTok{as.matrix}\NormalTok{(data[}\DecValTok{1}\OperatorTok{:}\DecValTok{270}\NormalTok{, }\DecValTok{14}\NormalTok{])}
\NormalTok{y_all <-}\StringTok{ }\KeywordTok{rep}\NormalTok{(}\DecValTok{0}\NormalTok{,}\DecValTok{270}\NormalTok{)}
\ControlFlowTok{for}\NormalTok{ (count }\ControlFlowTok{in} \DecValTok{1}\OperatorTok{:}\KeywordTok{length}\NormalTok{(class)) \{}
  \ControlFlowTok{if}\NormalTok{ (class[count] }\OperatorTok{==}\StringTok{ }\DecValTok{2}\NormalTok{ )\{}
\NormalTok{    y_all[count] =}\StringTok{ }\DecValTok{-1}
\NormalTok{  \}}
  \ControlFlowTok{else} \ControlFlowTok{if}\NormalTok{(class[count] }\OperatorTok{==}\StringTok{ }\DecValTok{1}\NormalTok{)\{}
\NormalTok{    y_all[count] =}\StringTok{ }\DecValTok{1}
\NormalTok{  \}}
\NormalTok{\}}

\ControlFlowTok{for}\NormalTok{ (p }\ControlFlowTok{in} \KeywordTok{c}\NormalTok{(}\DecValTok{5}\NormalTok{,}\DecValTok{10}\NormalTok{,}\DecValTok{30}\NormalTok{,}\DecValTok{50}\NormalTok{,}\DecValTok{100}\NormalTok{, }\DecValTok{150}\NormalTok{, }\DecValTok{200}\NormalTok{, }\DecValTok{300}\NormalTok{))\{}
  \CommentTok{# Realiza pelo 20 execuções diferentes}
\NormalTok{  accuracy_train <-}\StringTok{ }\KeywordTok{rep}\NormalTok{(}\DecValTok{0}\NormalTok{, }\DecValTok{20}\NormalTok{)}
\NormalTok{  accuracy_test <-}\StringTok{ }\KeywordTok{rep}\NormalTok{(}\DecValTok{0}\NormalTok{, }\DecValTok{20}\NormalTok{)}
  \ControlFlowTok{for}\NormalTok{(execution }\ControlFlowTok{in} \DecValTok{1}\OperatorTok{:}\DecValTok{20}\NormalTok{)\{}
    \CommentTok{# Separando dados entre treino e teste aleatoriamente:}
\NormalTok{    positions_train <-}\StringTok{ }\KeywordTok{createDataPartition}\NormalTok{(}\DecValTok{1}\OperatorTok{:}\DecValTok{270}\NormalTok{,}\DataTypeTok{p=}\NormalTok{.}\DecValTok{7}\NormalTok{)}
\NormalTok{    length_train <-}\StringTok{ }\KeywordTok{length}\NormalTok{(positions_train}\OperatorTok{$}\NormalTok{Resample1)}
\NormalTok{    length_test <-}\StringTok{ }\KeywordTok{length}\NormalTok{(y_all) }\OperatorTok{-}\StringTok{ }\NormalTok{length_train}
\NormalTok{    x_train <-}\StringTok{ }\KeywordTok{matrix}\NormalTok{(}\KeywordTok{rep}\NormalTok{(}\DecValTok{0}\NormalTok{, }\DecValTok{13}\OperatorTok{*}\NormalTok{length_train), }\DataTypeTok{ncol=}\DecValTok{13}\NormalTok{, }\DataTypeTok{nrow=}\NormalTok{length_train)}
\NormalTok{    y_train <-}\StringTok{ }\KeywordTok{rep}\NormalTok{(}\DecValTok{0}\NormalTok{, length_train)}
\NormalTok{    x_test <-}\StringTok{ }\KeywordTok{matrix}\NormalTok{(}\KeywordTok{rep}\NormalTok{(}\DecValTok{0}\NormalTok{, (}\DecValTok{13}\OperatorTok{*}\NormalTok{length_test)), }\DataTypeTok{ncol=}\DecValTok{13}\NormalTok{, }\DataTypeTok{nrow=}\NormalTok{(}\KeywordTok{length}\NormalTok{(y_all) }\OperatorTok{-}\StringTok{ }\NormalTok{length_train))}
\NormalTok{    y_test <-}\StringTok{ }\KeywordTok{rep}\NormalTok{(}\DecValTok{0}\NormalTok{, (}\KeywordTok{length}\NormalTok{(y_all) }\OperatorTok{-}\StringTok{ }\NormalTok{length_train))}
\NormalTok{    index_train <-}\StringTok{ }\DecValTok{1}
\NormalTok{    index_test <-}\StringTok{ }\DecValTok{1}
    \ControlFlowTok{for}\NormalTok{ (count }\ControlFlowTok{in} \DecValTok{1}\OperatorTok{:}\KeywordTok{length}\NormalTok{(y_all)) \{}
      \ControlFlowTok{if}\NormalTok{ (index_train }\OperatorTok{<=}\StringTok{ }\NormalTok{length_train }\OperatorTok{&&}\StringTok{ }\NormalTok{count }\OperatorTok{==}\StringTok{ }\NormalTok{positions_train}\OperatorTok{$}\NormalTok{Resample1[index_train])\{}
\NormalTok{        x_train[index_train, ] <-}\StringTok{ }\NormalTok{x_all[count, }\DecValTok{1}\OperatorTok{:}\DecValTok{13}\NormalTok{]}
\NormalTok{        y_train[index_train] <-}\StringTok{ }\NormalTok{y_all[count]}
\NormalTok{        index_train =}\StringTok{ }\NormalTok{index_train }\OperatorTok{+}\StringTok{ }\DecValTok{1}
\NormalTok{      \} }\ControlFlowTok{else}\NormalTok{ \{}
\NormalTok{        x_test[index_test, ] <-}\StringTok{ }\NormalTok{x_all[count, }\DecValTok{1}\OperatorTok{:}\DecValTok{13}\NormalTok{]}
\NormalTok{        y_test[index_test] <-}\StringTok{ }\NormalTok{y_all[count]}
\NormalTok{        index_test =}\StringTok{ }\NormalTok{index_test }\OperatorTok{+}\StringTok{ }\DecValTok{1}    
\NormalTok{      \}}
\NormalTok{    \}}
    
    \CommentTok{# Treinando modelo:}
\NormalTok{    retlist<-}\KeywordTok{trainELM}\NormalTok{(x_train, y_train, p, }\DecValTok{1}\NormalTok{)}
\NormalTok{    W<-retlist[[}\DecValTok{1}\NormalTok{]]}
\NormalTok{    H<-retlist[[}\DecValTok{2}\NormalTok{]]}
\NormalTok{    Z<-retlist[[}\DecValTok{3}\NormalTok{]]}
    
    \CommentTok{# Calculando acurácia de treinamento}
\NormalTok{    y_hat_train <-}\StringTok{ }\KeywordTok{as.matrix}\NormalTok{(}\KeywordTok{YELM}\NormalTok{(x_train, Z, W, }\DecValTok{1}\NormalTok{), }\DataTypeTok{nrow =}\NormalTok{ length_train, }\DataTypeTok{ncol =} \DecValTok{1}\NormalTok{)}
\NormalTok{    accuracy_train[execution]<-((}\KeywordTok{sum}\NormalTok{(}\KeywordTok{abs}\NormalTok{(y_hat_train }\OperatorTok{+}\StringTok{ }\NormalTok{y_train)))}\OperatorTok{/}\DecValTok{2}\NormalTok{)}\OperatorTok{/}\NormalTok{length_train}
    \CommentTok{#print(paste("Acurácia de treinamento para execução", execution, "com", p, "nerônios:", accuracy_train))}
    
    \CommentTok{# Calculando acurácia de Teste:}
\NormalTok{    y_hat_test <-}\StringTok{ }\KeywordTok{as.matrix}\NormalTok{(}\KeywordTok{YELM}\NormalTok{(x_test, Z, W, }\DecValTok{1}\NormalTok{), }\DataTypeTok{nrow =}\NormalTok{ length_test, }\DataTypeTok{ncol =} \DecValTok{1}\NormalTok{)}
\NormalTok{    accuracy_test[execution]<-((}\KeywordTok{sum}\NormalTok{(}\KeywordTok{abs}\NormalTok{(y_hat_test }\OperatorTok{+}\StringTok{ }\NormalTok{y_test)))}\OperatorTok{/}\DecValTok{2}\NormalTok{)}\OperatorTok{/}\NormalTok{length_test}
    \CommentTok{#print(paste("Acurácia de teste para execução", execution, "com", p, "nerônios:", accuracy_test))}
\NormalTok{  \}}
  \CommentTok{# Média das acurácias}
\NormalTok{  mean_accuracy_train <-}\StringTok{ }\KeywordTok{mean}\NormalTok{(accuracy_train) }\OperatorTok{*}\StringTok{ }\DecValTok{100}
\NormalTok{  mean_accuracy_test <-}\StringTok{ }\KeywordTok{mean}\NormalTok{(accuracy_test) }\OperatorTok{*}\StringTok{ }\DecValTok{100}
  
  \CommentTok{# Desvio Padrão das acurácias}
\NormalTok{  sd_accuracy_train <-}\StringTok{ }\KeywordTok{sd}\NormalTok{(accuracy_train) }\OperatorTok{*}\StringTok{ }\DecValTok{100}
\NormalTok{  sd_accuracy_test <-}\StringTok{ }\KeywordTok{sd}\NormalTok{(accuracy_test) }\OperatorTok{*}\StringTok{ }\DecValTok{100}
  
  \KeywordTok{print}\NormalTok{(}\KeywordTok{paste}\NormalTok{(}\StringTok{"Acurácia de treinamento do modelo com"}\NormalTok{, p, }\StringTok{"neurônios:"}\NormalTok{, mean_accuracy_train, }\StringTok{"%"}\NormalTok{, }\StringTok{"±"}\NormalTok{, sd_accuracy_train, }\StringTok{"%"}\NormalTok{))}
  \KeywordTok{print}\NormalTok{(}\KeywordTok{paste}\NormalTok{(}\StringTok{"Acurácia de teste do modelo com"}\NormalTok{, p, }\StringTok{"neurônios:"}\NormalTok{, mean_accuracy_test, }\StringTok{"%"}\NormalTok{, }\StringTok{"±"}\NormalTok{, sd_accuracy_test, }\StringTok{"%"}\NormalTok{))}
\NormalTok{\}}
\end{Highlighting}
\end{Shaded}

\begin{verbatim}
## [1] "Acurácia de treinamento do modelo com 5 neurônios: 59.5526315789474 % ± 4.46357012155771 %"
## [1] "Acurácia de teste do modelo com 5 neurônios: 55.9375 % ± 6.86901653498481 %"
## [1] "Acurácia de treinamento do modelo com 10 neurônios: 60.2631578947368 % ± 3.6802511371802 %"
## [1] "Acurácia de teste do modelo com 10 neurônios: 60.0625 % ± 6.31512126399386 %"
## [1] "Acurácia de treinamento do modelo com 30 neurônios: 67.2105263157895 % ± 3.64682255791487 %"
## [1] "Acurácia de teste do modelo com 30 neurônios: 63.125 % ± 5.24874671757584 %"
## [1] "Acurácia de treinamento do modelo com 50 neurônios: 69.3947368421053 % ± 3.44610473550686 %"
## [1] "Acurácia de teste do modelo com 50 neurônios: 62.1875 % ± 4.28957963220787 %"
## [1] "Acurácia de treinamento do modelo com 100 neurônios: 73.7894736842105 % ± 2.40643545990445 %"
## [1] "Acurácia de teste do modelo com 100 neurônios: 67.9375 % ± 5.85284175242654 %"
## [1] "Acurácia de treinamento do modelo com 150 neurônios: 78.0526315789474 % ± 3.83771995778915 %"
## [1] "Acurácia de teste do modelo com 150 neurônios: 63.5 % ± 5.71528421727306 %"
## [1] "Acurácia de treinamento do modelo com 200 neurônios: 81.5526315789474 % ± 2.66994197942477 %"
## [1] "Acurácia de teste do modelo com 200 neurônios: 61.75 % ± 6.63176169338847 %"
## [1] "Acurácia de treinamento do modelo com 300 neurônios: 86.1052631578947 % ± 2.48277342346912 %"
## [1] "Acurácia de teste do modelo com 300 neurônios: 61.9375 % ± 4.02449326656557 %"
\end{verbatim}

\hypertarget{discussuxe3o-2}{%
\paragraph{\texorpdfstring{\textbf{\emph{Discussão:}}}{Discussão:}}\label{discussuxe3o-2}}

Novamente pôde-se perceber que a acurácia de treinamento vai aumentando
progressivamente a medida que aumenta-se o número de neurônios. Contudo,
o mesmo não ocorre para a acurácia de teste. Como citado, uma explicação
plausível para isso é que um modelo com um número muito elevado de
neurônios pode se tornar super ajustado aos dados de treinamento e não
funcionar da maneira esperada com dados de teste. Aparentemente, a
acurácia média máxima de teste foi obtida para o número de 50 neurônios,
um valor menor que o número de neurônios para acurácia máxima da base de
dados anterior. Isso, pode ser explicado pela diferença de complexidade
dos problemas, enquanto na classificação do tumor verificou-se um número
de 30 features, nesse caso verificou-se apenas 13.

\hypertarget{perceptron-simples-com-base-de-dados-statlog-heart}{%
\subsection{Perceptron simples com base de dados Statlog
(Heart)}\label{perceptron-simples-com-base-de-dados-statlog-heart}}

Nessa etapa foi treinado um perceptron simples, utilizando a rotina de
treinamento de perceptron mostrada no início desse documento e a base de
dados Statlog (Heart), para também indicar a presença ou ausência de
problemas de coração. Foram utilizados 70\% dos dados para treino e 30\%
para teste. Além disso, foram realizados 20 execuções diferentes de
treinamento e teste, e os valores de acurácia (de treinamento e teste),
foram apresentados na forma de média ± desvio\_padrão. Os resultados e o
script desenvolvido podem ser vistos abaixo.

\begin{Shaded}
\begin{Highlighting}[]
\KeywordTok{rm}\NormalTok{(}\DataTypeTok{list=}\KeywordTok{ls}\NormalTok{())}
\KeywordTok{source}\NormalTok{(}\StringTok{"~/Documents/UFMG/9/Redes Neurais/listas/lista 4/trainPerceptron.R"}\NormalTok{)}
\KeywordTok{source}\NormalTok{(}\StringTok{"~/Documents/UFMG/9/Redes Neurais/listas/lista 4/yperceptron.R"}\NormalTok{)}
\KeywordTok{library}\NormalTok{(caret)}

\CommentTok{# Carregando base de dados:}
\NormalTok{path <-}\StringTok{ }\KeywordTok{file.path}\NormalTok{(}\StringTok{"~/Documents/UFMG/9/Redes Neurais/listas/lista 6/heart"}\NormalTok{, }\StringTok{"heart.csv"}\NormalTok{)}
\NormalTok{data <-}\StringTok{ }\KeywordTok{read.csv}\NormalTok{(path)}

\CommentTok{# Separando dados de entrada e saída:}
\NormalTok{x_all <-}\StringTok{ }\KeywordTok{as.matrix}\NormalTok{(data[}\DecValTok{1}\OperatorTok{:}\DecValTok{270}\NormalTok{, }\DecValTok{1}\OperatorTok{:}\DecValTok{13}\NormalTok{])}
\NormalTok{class <-}\StringTok{ }\KeywordTok{as.matrix}\NormalTok{(data[}\DecValTok{1}\OperatorTok{:}\DecValTok{270}\NormalTok{, }\DecValTok{14}\NormalTok{])}
\NormalTok{y_all <-}\StringTok{ }\KeywordTok{rep}\NormalTok{(}\DecValTok{0}\NormalTok{,}\DecValTok{270}\NormalTok{)}
\ControlFlowTok{for}\NormalTok{ (count }\ControlFlowTok{in} \DecValTok{1}\OperatorTok{:}\KeywordTok{length}\NormalTok{(class)) \{}
  \ControlFlowTok{if}\NormalTok{ (class[count] }\OperatorTok{==}\StringTok{ }\DecValTok{1}\NormalTok{ )\{}
\NormalTok{    y_all[count] =}\StringTok{ }\DecValTok{1}
\NormalTok{  \}}
  \ControlFlowTok{else} \ControlFlowTok{if}\NormalTok{(class[count] }\OperatorTok{==}\StringTok{ }\DecValTok{2}\NormalTok{)\{}
\NormalTok{    y_all[count] =}\StringTok{ }\DecValTok{0}
\NormalTok{  \}}
\NormalTok{\}}

\CommentTok{# Realiza pelo 20 execuções diferentes}
\NormalTok{accuracy_train <-}\StringTok{ }\KeywordTok{rep}\NormalTok{(}\DecValTok{0}\NormalTok{, }\DecValTok{20}\NormalTok{)}
\NormalTok{accuracy_test <-}\StringTok{ }\KeywordTok{rep}\NormalTok{(}\DecValTok{0}\NormalTok{, }\DecValTok{20}\NormalTok{)}
\ControlFlowTok{for}\NormalTok{(execution }\ControlFlowTok{in} \DecValTok{1}\OperatorTok{:}\DecValTok{20}\NormalTok{)\{}
  \CommentTok{# Separando dados entre treino e teste aleatoriamente:}
\NormalTok{  positions_train <-}\StringTok{ }\KeywordTok{createDataPartition}\NormalTok{(}\DecValTok{1}\OperatorTok{:}\DecValTok{270}\NormalTok{,}\DataTypeTok{p=}\NormalTok{.}\DecValTok{7}\NormalTok{)}
\NormalTok{  length_train <-}\StringTok{ }\KeywordTok{length}\NormalTok{(positions_train}\OperatorTok{$}\NormalTok{Resample1)}
\NormalTok{  length_test <-}\StringTok{ }\KeywordTok{length}\NormalTok{(y_all) }\OperatorTok{-}\StringTok{ }\NormalTok{length_train}
\NormalTok{  x_train <-}\StringTok{ }\KeywordTok{matrix}\NormalTok{(}\KeywordTok{rep}\NormalTok{(}\DecValTok{0}\NormalTok{, }\DecValTok{13}\OperatorTok{*}\NormalTok{length_train), }\DataTypeTok{ncol=}\DecValTok{13}\NormalTok{, }\DataTypeTok{nrow=}\NormalTok{length_train)}
\NormalTok{  y_train <-}\StringTok{ }\KeywordTok{rep}\NormalTok{(}\DecValTok{0}\NormalTok{, length_train)}
\NormalTok{  x_test <-}\StringTok{ }\KeywordTok{matrix}\NormalTok{(}\KeywordTok{rep}\NormalTok{(}\DecValTok{0}\NormalTok{, (}\DecValTok{13}\OperatorTok{*}\NormalTok{length_test)), }\DataTypeTok{ncol=}\DecValTok{13}\NormalTok{, }\DataTypeTok{nrow=}\NormalTok{(}\KeywordTok{length}\NormalTok{(y_all) }\OperatorTok{-}\StringTok{ }\NormalTok{length_train))}
\NormalTok{  y_test <-}\StringTok{ }\KeywordTok{rep}\NormalTok{(}\DecValTok{0}\NormalTok{, (}\KeywordTok{length}\NormalTok{(y_all) }\OperatorTok{-}\StringTok{ }\NormalTok{length_train))}
\NormalTok{  index_train <-}\StringTok{ }\DecValTok{1}
\NormalTok{  index_test <-}\StringTok{ }\DecValTok{1}
  \ControlFlowTok{for}\NormalTok{ (count }\ControlFlowTok{in} \DecValTok{1}\OperatorTok{:}\KeywordTok{length}\NormalTok{(y_all)) \{}
    \ControlFlowTok{if}\NormalTok{ (index_train }\OperatorTok{<=}\StringTok{ }\NormalTok{length_train }\OperatorTok{&&}\StringTok{ }\NormalTok{count }\OperatorTok{==}\StringTok{ }\NormalTok{positions_train}\OperatorTok{$}\NormalTok{Resample1[index_train])\{}
\NormalTok{      x_train[index_train, ] <-}\StringTok{ }\NormalTok{x_all[count, }\DecValTok{1}\OperatorTok{:}\DecValTok{13}\NormalTok{]}
\NormalTok{      y_train[index_train] <-}\StringTok{ }\NormalTok{y_all[count]}
\NormalTok{      index_train =}\StringTok{ }\NormalTok{index_train }\OperatorTok{+}\StringTok{ }\DecValTok{1}
\NormalTok{    \} }\ControlFlowTok{else}\NormalTok{ \{}
\NormalTok{      x_test[index_test, ] <-}\StringTok{ }\NormalTok{x_all[count, }\DecValTok{1}\OperatorTok{:}\DecValTok{13}\NormalTok{]}
\NormalTok{      y_test[index_test] <-}\StringTok{ }\NormalTok{y_all[count]}
\NormalTok{      index_test =}\StringTok{ }\NormalTok{index_test }\OperatorTok{+}\StringTok{ }\DecValTok{1}    
\NormalTok{    \}}
\NormalTok{  \}}
  
  \CommentTok{# Treinando modelo:}
\NormalTok{  retlist<-}\KeywordTok{trainPerceptron}\NormalTok{(x_train, y_train, }\FloatTok{0.1}\NormalTok{, }\FloatTok{0.01}\NormalTok{, }\DecValTok{1000}\NormalTok{, }\DecValTok{1}\NormalTok{)}
\NormalTok{  W<-retlist[[}\DecValTok{1}\NormalTok{]]}
  
  \CommentTok{# Calculando acurácia de treinamento}
\NormalTok{  y_hat_train <-}\StringTok{ }\KeywordTok{as.matrix}\NormalTok{(}\KeywordTok{yperceptron}\NormalTok{(x_train, W, }\DecValTok{1}\NormalTok{), }\DataTypeTok{nrow =}\NormalTok{ length_train, }\DataTypeTok{ncol =} \DecValTok{1}\NormalTok{)}
\NormalTok{  accuracy_train[execution]<-}\DecValTok{1}\OperatorTok{-}\NormalTok{((}\KeywordTok{t}\NormalTok{(y_hat_train}\OperatorTok{-}\NormalTok{y_train) }\OperatorTok\StringTok{ }\NormalTok{(y_hat_train}\OperatorTok{-}\NormalTok{y_train))}\OperatorTok{/}\NormalTok{length_train)}
  \CommentTok{#print(paste("Acurácia de treinamento para execução", execution, "com", p, "nerônios:", accuracy_train))}
  
  \CommentTok{# Calculando acurácia de Teste:}
\NormalTok{  y_hat_test <-}\StringTok{ }\KeywordTok{as.matrix}\NormalTok{(}\KeywordTok{yperceptron}\NormalTok{(x_test, W, }\DecValTok{1}\NormalTok{), }\DataTypeTok{nrow =}\NormalTok{ length_test, }\DataTypeTok{ncol =} \DecValTok{1}\NormalTok{)}
\NormalTok{  accuracy_test[execution]<-}\DecValTok{1}\OperatorTok{-}\NormalTok{((}\KeywordTok{t}\NormalTok{(y_hat_test}\OperatorTok{-}\NormalTok{y_test) }\OperatorTok\StringTok{ }\NormalTok{(y_hat_test}\OperatorTok{-}\NormalTok{y_test))}\OperatorTok{/}\NormalTok{length_test)}
  \CommentTok{#print(paste("Acurácia de teste para execução", execution, "com", p, "nerônios:", accuracy_test))}
\NormalTok{\}}
\CommentTok{# Média das acurácias}
\NormalTok{mean_accuracy_train <-}\StringTok{ }\KeywordTok{mean}\NormalTok{(accuracy_train) }\OperatorTok{*}\StringTok{ }\DecValTok{100}
\NormalTok{mean_accuracy_test <-}\StringTok{ }\KeywordTok{mean}\NormalTok{(accuracy_test) }\OperatorTok{*}\StringTok{ }\DecValTok{100}

\CommentTok{# Desvio Padrão das acurácias}
\NormalTok{sd_accuracy_train <-}\StringTok{ }\KeywordTok{sd}\NormalTok{(accuracy_train) }\OperatorTok{*}\StringTok{ }\DecValTok{100}
\NormalTok{sd_accuracy_test <-}\StringTok{ }\KeywordTok{sd}\NormalTok{(accuracy_test) }\OperatorTok{*}\StringTok{ }\DecValTok{100}

\KeywordTok{print}\NormalTok{(}\KeywordTok{paste}\NormalTok{(}\StringTok{"Acurácia de treinamento do modelo com perceptron simples"}\NormalTok{, mean_accuracy_train, }\StringTok{"%"}\NormalTok{, }\StringTok{"±"}\NormalTok{, sd_accuracy_train, }\StringTok{"%"}\NormalTok{))}
\end{Highlighting}
\end{Shaded}

\begin{verbatim}
## [1] "Acurácia de treinamento do modelo com perceptron simples 73.1842105263158 % ± 11.345512724782 %"
\end{verbatim}

\begin{Shaded}
\begin{Highlighting}[]
\KeywordTok{print}\NormalTok{(}\KeywordTok{paste}\NormalTok{(}\StringTok{"Acurácia de teste do modelo com perceptron simples"}\NormalTok{, mean_accuracy_test, }\StringTok{"%"}\NormalTok{, }\StringTok{"±"}\NormalTok{, sd_accuracy_test, }\StringTok{"%"}\NormalTok{))}
\end{Highlighting}
\end{Shaded}

\begin{verbatim}
## [1] "Acurácia de teste do modelo com perceptron simples 71.4375 % ± 13.7844962013965 %"
\end{verbatim}

\hypertarget{discussuxe3o-3}{%
\paragraph{\texorpdfstring{\textbf{\emph{Discussão:}}}{Discussão:}}\label{discussuxe3o-3}}

Pôde-se perceber que obteve-se uma acurácia de treinamento de cerca de
75\%, assim como para a ELM com 150 neurônios. Contudo, obteve-se uma
acurácia de teste média também próxima dos 75\%, o que é melhor do que a
obtida para as ELMs (que no melhor caso obteve uma acurácia média de
65\%).

\hypertarget{testes-com-dados-escalonados}{%
\subsection{Testes com dados
escalonados}\label{testes-com-dados-escalonados}}

Como sugerido pelo enunciado, por questões de convergência, pode ser
interessante escalonar os valores dos atributos para que fiquem
restritos entre 0 e 1. Isso foi feito para todos os exercícios
anteriores, utilizando a função abaixo desenvolvida pelo autor. Os
resultados para cada um dos testes podem ser vistos nas subseções
seguintes.

\begin{Shaded}
\begin{Highlighting}[]
\CommentTok{# Função que recebe uma matriz e suas dimensões e retorna uma matriz}
\CommentTok{# de mesma dimensão porém com sua colunas escalonadas.}
\NormalTok{staggeringMatrix <-}\StringTok{ }\ControlFlowTok{function}\NormalTok{(matrix, rows, columns )\{}
\NormalTok{  staggeredMatrix <-}\StringTok{ }\KeywordTok{matrix}\NormalTok{(}\KeywordTok{rep}\NormalTok{(}\DecValTok{0}\NormalTok{, rows}\OperatorTok{*}\NormalTok{columns), }\DataTypeTok{ncol =}\NormalTok{ columns, }\DataTypeTok{nrow =}\NormalTok{ rows)}
  \CommentTok{# Escalonando dados:}
  \ControlFlowTok{for}\NormalTok{ (j }\ControlFlowTok{in} \DecValTok{1}\OperatorTok{:}\NormalTok{columns) \{}
    \ControlFlowTok{for}\NormalTok{ (i }\ControlFlowTok{in} \DecValTok{1}\OperatorTok{:}\NormalTok{rows) \{}
\NormalTok{      staggeredMatrix[i,j] <-}\StringTok{ }\NormalTok{(matrix[i,j] }\OperatorTok{-}\StringTok{ }\KeywordTok{min}\NormalTok{(matrix[,j])) }\OperatorTok{/}\StringTok{ }\NormalTok{(}\KeywordTok{max}\NormalTok{(matrix[,j]) }\OperatorTok{-}\StringTok{ }\KeywordTok{min}\NormalTok{(matrix[,j]))}
\NormalTok{    \}}
\NormalTok{  \}}
  \KeywordTok{return}\NormalTok{(staggeredMatrix)}
\NormalTok{\}}
\end{Highlighting}
\end{Shaded}

\hypertarget{elm-com-base-de-dados-breast-cancer-diagnostic---dados-escalonados}{%
\paragraph{\texorpdfstring{\textbf{\emph{ELM com base de dados Breast
Cancer (diagnostic) - Dados
escalonados}}}{ELM com base de dados Breast Cancer (diagnostic) - Dados escalonados}}\label{elm-com-base-de-dados-breast-cancer-diagnostic---dados-escalonados}}

\begin{Shaded}
\begin{Highlighting}[]
\KeywordTok{rm}\NormalTok{(}\DataTypeTok{list=}\KeywordTok{ls}\NormalTok{())}
\KeywordTok{source}\NormalTok{(}\StringTok{"~/Documents/UFMG/9/Redes Neurais/exemplos/trainELM.R"}\NormalTok{)}
\KeywordTok{source}\NormalTok{(}\StringTok{"~/Documents/UFMG/9/Redes Neurais/exemplos/YELM.R"}\NormalTok{)}
\KeywordTok{source}\NormalTok{(}\StringTok{"~/Documents/UFMG/9/Redes Neurais/exemplos/escalonamento_matrix.R"}\NormalTok{)}
\KeywordTok{library}\NormalTok{(caret)}

\CommentTok{# Carregando base de dados:}
\NormalTok{path <-}\StringTok{ }\KeywordTok{file.path}\NormalTok{(}\StringTok{"~/Documents/UFMG/9/Redes Neurais/listas/lista 6/cancer"}\NormalTok{, }\StringTok{"wdbc.csv"}\NormalTok{)}
\NormalTok{data <-}\StringTok{ }\KeywordTok{read.csv}\NormalTok{(path)}

\CommentTok{# Separando dados de entrada e saída:}
\NormalTok{x_all <-}\StringTok{ }\KeywordTok{as.matrix}\NormalTok{(data[}\DecValTok{1}\OperatorTok{:}\DecValTok{569}\NormalTok{, }\DecValTok{3}\OperatorTok{:}\DecValTok{32}\NormalTok{])}
\NormalTok{class <-}\StringTok{ }\KeywordTok{as.matrix}\NormalTok{(data[}\DecValTok{1}\OperatorTok{:}\DecValTok{569}\NormalTok{, }\DecValTok{2}\NormalTok{])}
\NormalTok{y_all <-}\StringTok{ }\KeywordTok{rep}\NormalTok{(}\DecValTok{0}\NormalTok{,}\DecValTok{569}\NormalTok{)}
\ControlFlowTok{for}\NormalTok{ (count }\ControlFlowTok{in} \DecValTok{1}\OperatorTok{:}\KeywordTok{length}\NormalTok{(class)) \{}
  \ControlFlowTok{if}\NormalTok{ (class[count] }\OperatorTok{==}\StringTok{ 'M'}\NormalTok{ )\{}
\NormalTok{    y_all[count] =}\StringTok{ }\DecValTok{-1}
\NormalTok{  \}}
  \ControlFlowTok{else} \ControlFlowTok{if}\NormalTok{(class[count] }\OperatorTok{==}\StringTok{ 'B'}\NormalTok{)\{}
\NormalTok{    y_all[count] =}\StringTok{ }\DecValTok{1}
\NormalTok{  \}}
\NormalTok{\}}

\CommentTok{# Escalonando os valores dos atributos para que fiquem restritos entre 0 e 1}
\NormalTok{x_all <-}\StringTok{ }\KeywordTok{staggeringMatrix}\NormalTok{(x_all, }\KeywordTok{nrow}\NormalTok{(x_all), }\KeywordTok{ncol}\NormalTok{(x_all))}

\ControlFlowTok{for}\NormalTok{ (p }\ControlFlowTok{in} \KeywordTok{c}\NormalTok{(}\DecValTok{5}\NormalTok{,}\DecValTok{10}\NormalTok{,}\DecValTok{30}\NormalTok{,}\DecValTok{50}\NormalTok{,}\DecValTok{100}\NormalTok{,}\DecValTok{150}\NormalTok{,}\DecValTok{200}\NormalTok{,}\DecValTok{300}\NormalTok{))\{}
  \CommentTok{# Realiza pelo 20 execuções diferentes}
\NormalTok{  accuracy_train <-}\StringTok{ }\KeywordTok{rep}\NormalTok{(}\DecValTok{0}\NormalTok{, }\DecValTok{20}\NormalTok{)}
\NormalTok{  accuracy_test <-}\StringTok{ }\KeywordTok{rep}\NormalTok{(}\DecValTok{0}\NormalTok{, }\DecValTok{20}\NormalTok{)}
  \ControlFlowTok{for}\NormalTok{(execution }\ControlFlowTok{in} \DecValTok{1}\OperatorTok{:}\DecValTok{20}\NormalTok{)\{}
    \CommentTok{# Separando dados entre treino e teste aleatoriamente:}
\NormalTok{    positions_train <-}\StringTok{ }\KeywordTok{createDataPartition}\NormalTok{(}\DecValTok{1}\OperatorTok{:}\DecValTok{569}\NormalTok{,}\DataTypeTok{p=}\NormalTok{.}\DecValTok{7}\NormalTok{)}
\NormalTok{    length_train <-}\StringTok{ }\KeywordTok{length}\NormalTok{(positions_train}\OperatorTok{$}\NormalTok{Resample1)}
\NormalTok{    length_test <-}\StringTok{ }\KeywordTok{length}\NormalTok{(y_all) }\OperatorTok{-}\StringTok{ }\NormalTok{length_train}
\NormalTok{    x_train <-}\StringTok{ }\KeywordTok{matrix}\NormalTok{(}\KeywordTok{rep}\NormalTok{(}\DecValTok{0}\NormalTok{, }\DecValTok{30}\OperatorTok{*}\NormalTok{length_train), }\DataTypeTok{ncol=}\DecValTok{30}\NormalTok{, }\DataTypeTok{nrow=}\NormalTok{length_train)}
\NormalTok{    y_train <-}\StringTok{ }\KeywordTok{rep}\NormalTok{(}\DecValTok{0}\NormalTok{, length_train)}
\NormalTok{    x_test <-}\StringTok{ }\KeywordTok{matrix}\NormalTok{(}\KeywordTok{rep}\NormalTok{(}\DecValTok{0}\NormalTok{, (}\DecValTok{30}\OperatorTok{*}\NormalTok{length_test)), }\DataTypeTok{ncol=}\DecValTok{30}\NormalTok{, }\DataTypeTok{nrow=}\NormalTok{(}\KeywordTok{length}\NormalTok{(y_all) }\OperatorTok{-}\StringTok{ }\NormalTok{length_train))}
\NormalTok{    y_test <-}\StringTok{ }\KeywordTok{rep}\NormalTok{(}\DecValTok{0}\NormalTok{, (}\KeywordTok{length}\NormalTok{(y_all) }\OperatorTok{-}\StringTok{ }\NormalTok{length_train))}
\NormalTok{    index_train <-}\StringTok{ }\DecValTok{1}
\NormalTok{    index_test <-}\StringTok{ }\DecValTok{1}
    \ControlFlowTok{for}\NormalTok{ (count }\ControlFlowTok{in} \DecValTok{1}\OperatorTok{:}\KeywordTok{length}\NormalTok{(y_all)) \{}
      \ControlFlowTok{if}\NormalTok{ (index_train }\OperatorTok{<=}\StringTok{ }\NormalTok{length_train }\OperatorTok{&&}\StringTok{ }\NormalTok{count }\OperatorTok{==}\StringTok{ }\NormalTok{positions_train}\OperatorTok{$}\NormalTok{Resample1[index_train])\{}
\NormalTok{        x_train[index_train, ] <-}\StringTok{ }\NormalTok{x_all[count, }\DecValTok{1}\OperatorTok{:}\DecValTok{30}\NormalTok{ ]}
\NormalTok{        y_train[index_train] <-}\StringTok{ }\NormalTok{y_all[count]}
\NormalTok{        index_train =}\StringTok{ }\NormalTok{index_train }\OperatorTok{+}\StringTok{ }\DecValTok{1}
\NormalTok{      \} }\ControlFlowTok{else}\NormalTok{ \{}
\NormalTok{        x_test[index_test, ] <-}\StringTok{ }\NormalTok{x_all[count, }\DecValTok{1}\OperatorTok{:}\DecValTok{30}\NormalTok{ ]}
\NormalTok{        y_test[index_test] <-}\StringTok{ }\NormalTok{y_all[count]}
\NormalTok{        index_test =}\StringTok{ }\NormalTok{index_test }\OperatorTok{+}\StringTok{ }\DecValTok{1}    
\NormalTok{      \}}
\NormalTok{    \}}
    
    \CommentTok{# Treinando modelo:}
\NormalTok{    retlist<-}\KeywordTok{trainELM}\NormalTok{(x_train, y_train, p, }\DecValTok{1}\NormalTok{)}
\NormalTok{    W<-retlist[[}\DecValTok{1}\NormalTok{]]}
\NormalTok{    H<-retlist[[}\DecValTok{2}\NormalTok{]]}
\NormalTok{    Z<-retlist[[}\DecValTok{3}\NormalTok{]]}
    
    \CommentTok{# Calculando acurácia de treinamento}
\NormalTok{    y_hat_train <-}\StringTok{ }\KeywordTok{as.matrix}\NormalTok{(}\KeywordTok{YELM}\NormalTok{(x_train, Z, W, }\DecValTok{1}\NormalTok{), }\DataTypeTok{nrow =}\NormalTok{ length_train, }\DataTypeTok{ncol =} \DecValTok{1}\NormalTok{)}
\NormalTok{    accuracy_train[execution]<-((}\KeywordTok{sum}\NormalTok{(}\KeywordTok{abs}\NormalTok{(y_hat_train }\OperatorTok{+}\StringTok{ }\NormalTok{y_train)))}\OperatorTok{/}\DecValTok{2}\NormalTok{)}\OperatorTok{/}\NormalTok{length_train}
    \CommentTok{#print(paste("Acurácia de treinamento para execução", execution, "com", p, "nerônios:", accuracy_train))}
    
    \CommentTok{# Calculando acurácia de Teste:}
\NormalTok{    y_hat_test <-}\StringTok{ }\KeywordTok{as.matrix}\NormalTok{(}\KeywordTok{YELM}\NormalTok{(x_test, Z, W, }\DecValTok{1}\NormalTok{), }\DataTypeTok{nrow =}\NormalTok{ length_test, }\DataTypeTok{ncol =} \DecValTok{1}\NormalTok{)}
\NormalTok{    accuracy_test[execution]<-((}\KeywordTok{sum}\NormalTok{(}\KeywordTok{abs}\NormalTok{(y_hat_test }\OperatorTok{+}\StringTok{ }\NormalTok{y_test)))}\OperatorTok{/}\DecValTok{2}\NormalTok{)}\OperatorTok{/}\NormalTok{length_test}
    \CommentTok{#print(paste("Acurácia de teste para execução", execution, "com", p, "nerônios:", accuracy_test))}
\NormalTok{  \}}
  \CommentTok{# Média das acurácias}
\NormalTok{  mean_accuracy_train <-}\StringTok{ }\KeywordTok{mean}\NormalTok{(accuracy_train) }\OperatorTok{*}\StringTok{ }\DecValTok{100}
\NormalTok{  mean_accuracy_test <-}\StringTok{ }\KeywordTok{mean}\NormalTok{(accuracy_test) }\OperatorTok{*}\StringTok{ }\DecValTok{100}
  
  \CommentTok{# Desvio Padrão das acurácias}
\NormalTok{  sd_accuracy_train <-}\StringTok{ }\KeywordTok{sd}\NormalTok{(accuracy_train) }\OperatorTok{*}\StringTok{ }\DecValTok{100}
\NormalTok{  sd_accuracy_test <-}\StringTok{ }\KeywordTok{sd}\NormalTok{(accuracy_test) }\OperatorTok{*}\StringTok{ }\DecValTok{100}
  
  \KeywordTok{print}\NormalTok{(}\KeywordTok{paste}\NormalTok{(}\StringTok{"Acurácia de treinamento do modelo com"}\NormalTok{, p, }\StringTok{"neurônios:"}\NormalTok{, mean_accuracy_train, }\StringTok{"%"}\NormalTok{, }\StringTok{"±"}\NormalTok{, sd_accuracy_train, }\StringTok{"%"}\NormalTok{))}
  \KeywordTok{print}\NormalTok{(}\KeywordTok{paste}\NormalTok{(}\StringTok{"Acurácia de teste do modelo com"}\NormalTok{, p, }\StringTok{"neurônios:"}\NormalTok{, mean_accuracy_test, }\StringTok{"%"}\NormalTok{, }\StringTok{"±"}\NormalTok{, sd_accuracy_test, }\StringTok{"%"}\NormalTok{))}
\NormalTok{\}}
\end{Highlighting}
\end{Shaded}

\begin{verbatim}
## [1] "Acurácia de treinamento do modelo com 5 neurônios: 77.6059850374065 % ± 5.31452059401984 %"
## [1] "Acurácia de teste do modelo com 5 neurônios: 77.1428571428572 % ± 5.24065783750072 %"
## [1] "Acurácia de treinamento do modelo com 10 neurônios: 85.5486284289277 % ± 3.88316260116539 %"
## [1] "Acurácia de teste do modelo com 10 neurônios: 85.625 % ± 3.41815525631118 %"
## [1] "Acurácia de treinamento do modelo com 30 neurônios: 94.5386533665835 % ± 0.915016861205572 %"
## [1] "Acurácia de teste do modelo com 30 neurônios: 92.9464285714286 % ± 1.68663078819345 %"
## [1] "Acurácia de treinamento do modelo com 50 neurônios: 95.9351620947631 % ± 0.600578283231536 %"
## [1] "Acurácia de teste do modelo com 50 neurônios: 93.8690476190476 % ± 1.49715288536056 %"
## [1] "Acurácia de treinamento do modelo com 100 neurônios: 97.2942643391521 % ± 0.466716179623811 %"
## [1] "Acurácia de teste do modelo com 100 neurônios: 91.6964285714286 % ± 1.75384043165188 %"
## [1] "Acurácia de treinamento do modelo com 150 neurônios: 97.9675810473815 % ± 0.729411008976228 %"
## [1] "Acurácia de teste do modelo com 150 neurônios: 88.8392857142857 % ± 2.95339010766581 %"
## [1] "Acurácia de treinamento do modelo com 200 neurônios: 98.6658354114713 % ± 0.480537556780449 %"
## [1] "Acurácia de teste do modelo com 200 neurônios: 83.8095238095238 % ± 3.08751687810935 %"
## [1] "Acurácia de treinamento do modelo com 300 neurônios: 99.8379052369077 % ± 0.185825391613613 %"
## [1] "Acurácia de teste do modelo com 300 neurônios: 72.1726190476191 % ± 3.85697944361285 %"
\end{verbatim}

\hypertarget{perceptron-simples-com-base-de-dados-breast-cancer-diagnostic---dados-escalonados}{%
\paragraph{\texorpdfstring{\textbf{\emph{Perceptron simples com base de
dados Breast Cancer (diagnostic) - Dados
escalonados}}}{Perceptron simples com base de dados Breast Cancer (diagnostic) - Dados escalonados}}\label{perceptron-simples-com-base-de-dados-breast-cancer-diagnostic---dados-escalonados}}

\begin{Shaded}
\begin{Highlighting}[]
\KeywordTok{rm}\NormalTok{(}\DataTypeTok{list=}\KeywordTok{ls}\NormalTok{())}
\KeywordTok{source}\NormalTok{(}\StringTok{"~/Documents/UFMG/9/Redes Neurais/listas/lista 4/trainPerceptron.R"}\NormalTok{)}
\KeywordTok{source}\NormalTok{(}\StringTok{"~/Documents/UFMG/9/Redes Neurais/listas/lista 4/yperceptron.R"}\NormalTok{)}
\KeywordTok{source}\NormalTok{(}\StringTok{"~/Documents/UFMG/9/Redes Neurais/exemplos/escalonamento_matrix.R"}\NormalTok{)}
\KeywordTok{library}\NormalTok{(caret)}

\CommentTok{# Carregando base de dados:}
\NormalTok{path <-}\StringTok{ }\KeywordTok{file.path}\NormalTok{(}\StringTok{"~/Documents/UFMG/9/Redes Neurais/listas/lista 6/cancer"}\NormalTok{, }\StringTok{"wdbc.csv"}\NormalTok{)}
\NormalTok{data <-}\StringTok{ }\KeywordTok{read.csv}\NormalTok{(path)}

\CommentTok{# Separando dados de entrada e saída:}
\NormalTok{x_all <-}\StringTok{ }\KeywordTok{as.matrix}\NormalTok{(data[}\DecValTok{1}\OperatorTok{:}\DecValTok{569}\NormalTok{, }\DecValTok{3}\OperatorTok{:}\DecValTok{32}\NormalTok{])}
\NormalTok{class <-}\StringTok{ }\KeywordTok{as.matrix}\NormalTok{(data[}\DecValTok{1}\OperatorTok{:}\DecValTok{569}\NormalTok{, }\DecValTok{2}\NormalTok{])}
\NormalTok{y_all <-}\StringTok{ }\KeywordTok{rep}\NormalTok{(}\DecValTok{0}\NormalTok{,}\DecValTok{569}\NormalTok{)}
\ControlFlowTok{for}\NormalTok{ (count }\ControlFlowTok{in} \DecValTok{1}\OperatorTok{:}\KeywordTok{length}\NormalTok{(class)) \{}
  \ControlFlowTok{if}\NormalTok{ (class[count] }\OperatorTok{==}\StringTok{ 'M'}\NormalTok{ )\{}
\NormalTok{    y_all[count] =}\StringTok{ }\DecValTok{0}
\NormalTok{  \}}
  \ControlFlowTok{else} \ControlFlowTok{if}\NormalTok{(class[count] }\OperatorTok{==}\StringTok{ 'B'}\NormalTok{)\{}
\NormalTok{    y_all[count] =}\StringTok{ }\DecValTok{1}
\NormalTok{  \}}
\NormalTok{\}}

\CommentTok{# Escalonando os valores dos atributos para que fiquem restritos entre 0 e 1}
\NormalTok{x_all <-}\StringTok{ }\KeywordTok{staggeringMatrix}\NormalTok{(x_all, }\KeywordTok{nrow}\NormalTok{(x_all), }\KeywordTok{ncol}\NormalTok{(x_all))}


\CommentTok{# Realiza pelo 20 execuções diferentes}
\NormalTok{accuracy_train <-}\StringTok{ }\KeywordTok{rep}\NormalTok{(}\DecValTok{0}\NormalTok{, }\DecValTok{20}\NormalTok{)}
\NormalTok{accuracy_test <-}\StringTok{ }\KeywordTok{rep}\NormalTok{(}\DecValTok{0}\NormalTok{, }\DecValTok{20}\NormalTok{)}
\ControlFlowTok{for}\NormalTok{(execution }\ControlFlowTok{in} \DecValTok{1}\OperatorTok{:}\DecValTok{20}\NormalTok{)\{}
  \CommentTok{# Separando dados entre treino e teste aleatoriamente:}
\NormalTok{  positions_train <-}\StringTok{ }\KeywordTok{createDataPartition}\NormalTok{(}\DecValTok{1}\OperatorTok{:}\DecValTok{569}\NormalTok{,}\DataTypeTok{p=}\NormalTok{.}\DecValTok{7}\NormalTok{)}
\NormalTok{  length_train <-}\StringTok{ }\KeywordTok{length}\NormalTok{(positions_train}\OperatorTok{$}\NormalTok{Resample1)}
\NormalTok{  length_test <-}\StringTok{ }\KeywordTok{length}\NormalTok{(y_all) }\OperatorTok{-}\StringTok{ }\NormalTok{length_train}
\NormalTok{  x_train <-}\StringTok{ }\KeywordTok{matrix}\NormalTok{(}\KeywordTok{rep}\NormalTok{(}\DecValTok{0}\NormalTok{, }\DecValTok{30}\OperatorTok{*}\NormalTok{length_train), }\DataTypeTok{ncol=}\DecValTok{30}\NormalTok{, }\DataTypeTok{nrow=}\NormalTok{length_train)}
\NormalTok{  y_train <-}\StringTok{ }\KeywordTok{rep}\NormalTok{(}\DecValTok{0}\NormalTok{, length_train)}
\NormalTok{  x_test <-}\StringTok{ }\KeywordTok{matrix}\NormalTok{(}\KeywordTok{rep}\NormalTok{(}\DecValTok{0}\NormalTok{, (}\DecValTok{30}\OperatorTok{*}\NormalTok{length_test)), }\DataTypeTok{ncol=}\DecValTok{30}\NormalTok{, }\DataTypeTok{nrow=}\NormalTok{(}\KeywordTok{length}\NormalTok{(y_all) }\OperatorTok{-}\StringTok{ }\NormalTok{length_train))}
\NormalTok{  y_test <-}\StringTok{ }\KeywordTok{rep}\NormalTok{(}\DecValTok{0}\NormalTok{, (}\KeywordTok{length}\NormalTok{(y_all) }\OperatorTok{-}\StringTok{ }\NormalTok{length_train))}
\NormalTok{  index_train <-}\StringTok{ }\DecValTok{1}
\NormalTok{  index_test <-}\StringTok{ }\DecValTok{1}
  \ControlFlowTok{for}\NormalTok{ (count }\ControlFlowTok{in} \DecValTok{1}\OperatorTok{:}\KeywordTok{length}\NormalTok{(y_all)) \{}
    \ControlFlowTok{if}\NormalTok{ (index_train }\OperatorTok{<=}\StringTok{ }\NormalTok{length_train }\OperatorTok{&&}\StringTok{ }\NormalTok{count }\OperatorTok{==}\StringTok{ }\NormalTok{positions_train}\OperatorTok{$}\NormalTok{Resample1[index_train])\{}
\NormalTok{      x_train[index_train, ] <-}\StringTok{ }\NormalTok{x_all[count, }\DecValTok{1}\OperatorTok{:}\DecValTok{30}\NormalTok{ ]}
\NormalTok{      y_train[index_train] <-}\StringTok{ }\NormalTok{y_all[count]}
\NormalTok{      index_train =}\StringTok{ }\NormalTok{index_train }\OperatorTok{+}\StringTok{ }\DecValTok{1}
\NormalTok{    \} }\ControlFlowTok{else}\NormalTok{ \{}
\NormalTok{      x_test[index_test, ] <-}\StringTok{ }\NormalTok{x_all[count, }\DecValTok{1}\OperatorTok{:}\DecValTok{30}\NormalTok{ ]}
\NormalTok{      y_test[index_test] <-}\StringTok{ }\NormalTok{y_all[count]}
\NormalTok{      index_test =}\StringTok{ }\NormalTok{index_test }\OperatorTok{+}\StringTok{ }\DecValTok{1}    
\NormalTok{    \}}
\NormalTok{  \}}
    
  \CommentTok{# Treinando modelo:}
\NormalTok{  retlist<-}\KeywordTok{trainPerceptron}\NormalTok{(x_train, y_train, }\FloatTok{0.1}\NormalTok{, }\FloatTok{0.01}\NormalTok{, }\DecValTok{1000}\NormalTok{, }\DecValTok{1}\NormalTok{)}
\NormalTok{  W<-retlist[[}\DecValTok{1}\NormalTok{]]}
    
  \CommentTok{# Calculando acurácia de treinamento}
\NormalTok{  y_hat_train <-}\StringTok{ }\KeywordTok{as.matrix}\NormalTok{(}\KeywordTok{yperceptron}\NormalTok{(x_train, W, }\DecValTok{1}\NormalTok{), }\DataTypeTok{nrow =}\NormalTok{ length_train, }\DataTypeTok{ncol =} \DecValTok{1}\NormalTok{)}
\NormalTok{  accuracy_train[execution]<-}\DecValTok{1}\OperatorTok{-}\NormalTok{((}\KeywordTok{t}\NormalTok{(y_hat_train}\OperatorTok{-}\NormalTok{y_train) }\OperatorTok\StringTok{ }\NormalTok{(y_hat_train}\OperatorTok{-}\NormalTok{y_train))}\OperatorTok{/}\NormalTok{length_train)}
  \CommentTok{#print(paste("Acurácia de treinamento para execução", execution, "com", p, "nerônios:", accuracy_train))}
    
  \CommentTok{# Calculando acurácia de Teste:}
\NormalTok{  y_hat_test <-}\StringTok{ }\KeywordTok{as.matrix}\NormalTok{(}\KeywordTok{yperceptron}\NormalTok{(x_test, W, }\DecValTok{1}\NormalTok{), }\DataTypeTok{nrow =}\NormalTok{ length_test, }\DataTypeTok{ncol =} \DecValTok{1}\NormalTok{)}
\NormalTok{  accuracy_test[execution]<-}\DecValTok{1}\OperatorTok{-}\NormalTok{((}\KeywordTok{t}\NormalTok{(y_hat_test}\OperatorTok{-}\NormalTok{y_test) }\OperatorTok\StringTok{ }\NormalTok{(y_hat_test}\OperatorTok{-}\NormalTok{y_test))}\OperatorTok{/}\NormalTok{length_test)}
  \CommentTok{#print(paste("Acurácia de teste para execução", execution, "com", p, "nerônios:", accuracy_test))}
\NormalTok{\}}
\CommentTok{# Média das acurácias}
\NormalTok{mean_accuracy_train <-}\StringTok{ }\KeywordTok{mean}\NormalTok{(accuracy_train) }\OperatorTok{*}\StringTok{ }\DecValTok{100}
\NormalTok{mean_accuracy_test <-}\StringTok{ }\KeywordTok{mean}\NormalTok{(accuracy_test) }\OperatorTok{*}\StringTok{ }\DecValTok{100}
  
\CommentTok{# Desvio Padrão das acurácias}
\NormalTok{sd_accuracy_train <-}\StringTok{ }\KeywordTok{sd}\NormalTok{(accuracy_train) }\OperatorTok{*}\StringTok{ }\DecValTok{100}
\NormalTok{sd_accuracy_test <-}\StringTok{ }\KeywordTok{sd}\NormalTok{(accuracy_test) }\OperatorTok{*}\StringTok{ }\DecValTok{100}
  
\KeywordTok{print}\NormalTok{(}\KeywordTok{paste}\NormalTok{(}\StringTok{"Acurácia de treinamento do modelo com perceptron simples"}\NormalTok{, mean_accuracy_train, }\StringTok{"%"}\NormalTok{, }\StringTok{"±"}\NormalTok{, sd_accuracy_train, }\StringTok{"%"}\NormalTok{))}
\end{Highlighting}
\end{Shaded}

\begin{verbatim}
## [1] "Acurácia de treinamento do modelo com perceptron simples 97.0573566084788 % ± 2.42131262886612 %"
\end{verbatim}

\begin{Shaded}
\begin{Highlighting}[]
\KeywordTok{print}\NormalTok{(}\KeywordTok{paste}\NormalTok{(}\StringTok{"Acurácia de teste do modelo com perceptron simples"}\NormalTok{, mean_accuracy_test, }\StringTok{"%"}\NormalTok{, }\StringTok{"±"}\NormalTok{, sd_accuracy_test, }\StringTok{"%"}\NormalTok{))}
\end{Highlighting}
\end{Shaded}

\begin{verbatim}
## [1] "Acurácia de teste do modelo com perceptron simples 94.1964285714286 % ± 2.27583594044726 %"
\end{verbatim}

\hypertarget{elm-com-base-de-dados-statlog-heart---dados-escalonados}{%
\paragraph{\texorpdfstring{\textbf{\emph{ELM com base de dados Statlog
(Heart) - Dados
escalonados}}}{ELM com base de dados Statlog (Heart) - Dados escalonados}}\label{elm-com-base-de-dados-statlog-heart---dados-escalonados}}

\begin{Shaded}
\begin{Highlighting}[]
\KeywordTok{rm}\NormalTok{(}\DataTypeTok{list=}\KeywordTok{ls}\NormalTok{())}
\KeywordTok{source}\NormalTok{(}\StringTok{"~/Documents/UFMG/9/Redes Neurais/exemplos/trainELM.R"}\NormalTok{)}
\KeywordTok{source}\NormalTok{(}\StringTok{"~/Documents/UFMG/9/Redes Neurais/exemplos/YELM.R"}\NormalTok{)}
\KeywordTok{source}\NormalTok{(}\StringTok{"~/Documents/UFMG/9/Redes Neurais/exemplos/escalonamento_matrix.R"}\NormalTok{)}
\KeywordTok{library}\NormalTok{(caret)}

\CommentTok{# Carregando base de dados:}
\NormalTok{path <-}\StringTok{ }\KeywordTok{file.path}\NormalTok{(}\StringTok{"~/Documents/UFMG/9/Redes Neurais/listas/lista 6/heart"}\NormalTok{, }\StringTok{"heart.csv"}\NormalTok{)}
\NormalTok{data <-}\StringTok{ }\KeywordTok{read.csv}\NormalTok{(path)}

\CommentTok{# Separando dados de entrada e saída:}
\NormalTok{x_all <-}\StringTok{ }\KeywordTok{as.matrix}\NormalTok{(data[}\DecValTok{1}\OperatorTok{:}\DecValTok{270}\NormalTok{, }\DecValTok{1}\OperatorTok{:}\DecValTok{13}\NormalTok{])}
\NormalTok{class <-}\StringTok{ }\KeywordTok{as.matrix}\NormalTok{(data[}\DecValTok{1}\OperatorTok{:}\DecValTok{270}\NormalTok{, }\DecValTok{14}\NormalTok{])}
\NormalTok{y_all <-}\StringTok{ }\KeywordTok{rep}\NormalTok{(}\DecValTok{0}\NormalTok{,}\DecValTok{270}\NormalTok{)}
\ControlFlowTok{for}\NormalTok{ (count }\ControlFlowTok{in} \DecValTok{1}\OperatorTok{:}\KeywordTok{length}\NormalTok{(class)) \{}
  \ControlFlowTok{if}\NormalTok{ (class[count] }\OperatorTok{==}\StringTok{ }\DecValTok{2}\NormalTok{ )\{}
\NormalTok{    y_all[count] =}\StringTok{ }\DecValTok{-1}
\NormalTok{  \}}
  \ControlFlowTok{else} \ControlFlowTok{if}\NormalTok{(class[count] }\OperatorTok{==}\StringTok{ }\DecValTok{1}\NormalTok{)\{}
\NormalTok{    y_all[count] =}\StringTok{ }\DecValTok{1}
\NormalTok{  \}}
\NormalTok{\}}

\CommentTok{# Escalonando os valores dos atributos para que fiquem restritos entre 0 e 1}
\NormalTok{x_all <-}\StringTok{ }\KeywordTok{staggeringMatrix}\NormalTok{(x_all, }\KeywordTok{nrow}\NormalTok{(x_all), }\KeywordTok{ncol}\NormalTok{(x_all))}

\ControlFlowTok{for}\NormalTok{ (p }\ControlFlowTok{in} \KeywordTok{c}\NormalTok{(}\DecValTok{5}\NormalTok{,}\DecValTok{10}\NormalTok{,}\DecValTok{30}\NormalTok{,}\DecValTok{50}\NormalTok{,}\DecValTok{100}\NormalTok{, }\DecValTok{150}\NormalTok{, }\DecValTok{200}\NormalTok{, }\DecValTok{300}\NormalTok{))\{}
  \CommentTok{# Realiza pelo 20 execuções diferentes}
\NormalTok{  accuracy_train <-}\StringTok{ }\KeywordTok{rep}\NormalTok{(}\DecValTok{0}\NormalTok{, }\DecValTok{20}\NormalTok{)}
\NormalTok{  accuracy_test <-}\StringTok{ }\KeywordTok{rep}\NormalTok{(}\DecValTok{0}\NormalTok{, }\DecValTok{20}\NormalTok{)}
  \ControlFlowTok{for}\NormalTok{(execution }\ControlFlowTok{in} \DecValTok{1}\OperatorTok{:}\DecValTok{20}\NormalTok{)\{}
    \CommentTok{# Separando dados entre treino e teste aleatoriamente:}
\NormalTok{    positions_train <-}\StringTok{ }\KeywordTok{createDataPartition}\NormalTok{(}\DecValTok{1}\OperatorTok{:}\DecValTok{270}\NormalTok{,}\DataTypeTok{p=}\NormalTok{.}\DecValTok{7}\NormalTok{)}
\NormalTok{    length_train <-}\StringTok{ }\KeywordTok{length}\NormalTok{(positions_train}\OperatorTok{$}\NormalTok{Resample1)}
\NormalTok{    length_test <-}\StringTok{ }\KeywordTok{length}\NormalTok{(y_all) }\OperatorTok{-}\StringTok{ }\NormalTok{length_train}
\NormalTok{    x_train <-}\StringTok{ }\KeywordTok{matrix}\NormalTok{(}\KeywordTok{rep}\NormalTok{(}\DecValTok{0}\NormalTok{, }\DecValTok{13}\OperatorTok{*}\NormalTok{length_train), }\DataTypeTok{ncol=}\DecValTok{13}\NormalTok{, }\DataTypeTok{nrow=}\NormalTok{length_train)}
\NormalTok{    y_train <-}\StringTok{ }\KeywordTok{rep}\NormalTok{(}\DecValTok{0}\NormalTok{, length_train)}
\NormalTok{    x_test <-}\StringTok{ }\KeywordTok{matrix}\NormalTok{(}\KeywordTok{rep}\NormalTok{(}\DecValTok{0}\NormalTok{, (}\DecValTok{13}\OperatorTok{*}\NormalTok{length_test)), }\DataTypeTok{ncol=}\DecValTok{13}\NormalTok{, }\DataTypeTok{nrow=}\NormalTok{(}\KeywordTok{length}\NormalTok{(y_all) }\OperatorTok{-}\StringTok{ }\NormalTok{length_train))}
\NormalTok{    y_test <-}\StringTok{ }\KeywordTok{rep}\NormalTok{(}\DecValTok{0}\NormalTok{, (}\KeywordTok{length}\NormalTok{(y_all) }\OperatorTok{-}\StringTok{ }\NormalTok{length_train))}
\NormalTok{    index_train <-}\StringTok{ }\DecValTok{1}
\NormalTok{    index_test <-}\StringTok{ }\DecValTok{1}
    \ControlFlowTok{for}\NormalTok{ (count }\ControlFlowTok{in} \DecValTok{1}\OperatorTok{:}\KeywordTok{length}\NormalTok{(y_all)) \{}
      \ControlFlowTok{if}\NormalTok{ (index_train }\OperatorTok{<=}\StringTok{ }\NormalTok{length_train }\OperatorTok{&&}\StringTok{ }\NormalTok{count }\OperatorTok{==}\StringTok{ }\NormalTok{positions_train}\OperatorTok{$}\NormalTok{Resample1[index_train])\{}
\NormalTok{        x_train[index_train, ] <-}\StringTok{ }\NormalTok{x_all[count, }\DecValTok{1}\OperatorTok{:}\DecValTok{13}\NormalTok{]}
\NormalTok{        y_train[index_train] <-}\StringTok{ }\NormalTok{y_all[count]}
\NormalTok{        index_train =}\StringTok{ }\NormalTok{index_train }\OperatorTok{+}\StringTok{ }\DecValTok{1}
\NormalTok{      \} }\ControlFlowTok{else}\NormalTok{ \{}
\NormalTok{        x_test[index_test, ] <-}\StringTok{ }\NormalTok{x_all[count, }\DecValTok{1}\OperatorTok{:}\DecValTok{13}\NormalTok{]}
\NormalTok{        y_test[index_test] <-}\StringTok{ }\NormalTok{y_all[count]}
\NormalTok{        index_test =}\StringTok{ }\NormalTok{index_test }\OperatorTok{+}\StringTok{ }\DecValTok{1}    
\NormalTok{      \}}
\NormalTok{    \}}
    
    \CommentTok{# Treinando modelo:}
\NormalTok{    retlist<-}\KeywordTok{trainELM}\NormalTok{(x_train, y_train, p, }\DecValTok{1}\NormalTok{)}
\NormalTok{    W<-retlist[[}\DecValTok{1}\NormalTok{]]}
\NormalTok{    H<-retlist[[}\DecValTok{2}\NormalTok{]]}
\NormalTok{    Z<-retlist[[}\DecValTok{3}\NormalTok{]]}
    
    \CommentTok{# Calculando acurácia de treinamento}
\NormalTok{    y_hat_train <-}\StringTok{ }\KeywordTok{as.matrix}\NormalTok{(}\KeywordTok{YELM}\NormalTok{(x_train, Z, W, }\DecValTok{1}\NormalTok{), }\DataTypeTok{nrow =}\NormalTok{ length_train, }\DataTypeTok{ncol =} \DecValTok{1}\NormalTok{)}
\NormalTok{    accuracy_train[execution]<-((}\KeywordTok{sum}\NormalTok{(}\KeywordTok{abs}\NormalTok{(y_hat_train }\OperatorTok{+}\StringTok{ }\NormalTok{y_train)))}\OperatorTok{/}\DecValTok{2}\NormalTok{)}\OperatorTok{/}\NormalTok{length_train}
    \CommentTok{#print(paste("Acurácia de treinamento para execução", execution, "com", p, "nerônios:", accuracy_train))}
    
    \CommentTok{# Calculando acurácia de Teste:}
\NormalTok{    y_hat_test <-}\StringTok{ }\KeywordTok{as.matrix}\NormalTok{(}\KeywordTok{YELM}\NormalTok{(x_test, Z, W, }\DecValTok{1}\NormalTok{), }\DataTypeTok{nrow =}\NormalTok{ length_test, }\DataTypeTok{ncol =} \DecValTok{1}\NormalTok{)}
\NormalTok{    accuracy_test[execution]<-((}\KeywordTok{sum}\NormalTok{(}\KeywordTok{abs}\NormalTok{(y_hat_test }\OperatorTok{+}\StringTok{ }\NormalTok{y_test)))}\OperatorTok{/}\DecValTok{2}\NormalTok{)}\OperatorTok{/}\NormalTok{length_test}
    \CommentTok{#print(paste("Acurácia de teste para execução", execution, "com", p, "nerônios:", accuracy_test))}
\NormalTok{  \}}
  \CommentTok{# Média das acurácias}
\NormalTok{  mean_accuracy_train <-}\StringTok{ }\KeywordTok{mean}\NormalTok{(accuracy_train) }\OperatorTok{*}\StringTok{ }\DecValTok{100}
\NormalTok{  mean_accuracy_test <-}\StringTok{ }\KeywordTok{mean}\NormalTok{(accuracy_test) }\OperatorTok{*}\StringTok{ }\DecValTok{100}
  
  \CommentTok{# Desvio Padrão das acurácias}
\NormalTok{  sd_accuracy_train <-}\StringTok{ }\KeywordTok{sd}\NormalTok{(accuracy_train) }\OperatorTok{*}\StringTok{ }\DecValTok{100}
\NormalTok{  sd_accuracy_test <-}\StringTok{ }\KeywordTok{sd}\NormalTok{(accuracy_test) }\OperatorTok{*}\StringTok{ }\DecValTok{100}
  
  \KeywordTok{print}\NormalTok{(}\KeywordTok{paste}\NormalTok{(}\StringTok{"Acurácia de treinamento do modelo com"}\NormalTok{, p, }\StringTok{"neurônios:"}\NormalTok{, mean_accuracy_train, }\StringTok{"%"}\NormalTok{, }\StringTok{"±"}\NormalTok{, sd_accuracy_train, }\StringTok{"%"}\NormalTok{))}
  \KeywordTok{print}\NormalTok{(}\KeywordTok{paste}\NormalTok{(}\StringTok{"Acurácia de teste do modelo com"}\NormalTok{, p, }\StringTok{"neurônios:"}\NormalTok{, mean_accuracy_test, }\StringTok{"%"}\NormalTok{, }\StringTok{"±"}\NormalTok{, sd_accuracy_test, }\StringTok{"%"}\NormalTok{))}
\NormalTok{\}}
\end{Highlighting}
\end{Shaded}

\begin{verbatim}
## [1] "Acurácia de treinamento do modelo com 5 neurônios: 75.3421052631579 % ± 4.79428086656957 %"
## [1] "Acurácia de teste do modelo com 5 neurônios: 73.75 % ± 6.16388035677459 %"
## [1] "Acurácia de treinamento do modelo com 10 neurônios: 81.7105263157895 % ± 3.3456423362016 %"
## [1] "Acurácia de teste do modelo com 10 neurônios: 79.625 % ± 3.8708594176977 %"
## [1] "Acurácia de treinamento do modelo com 30 neurônios: 87.3947368421053 % ± 1.47363474392642 %"
## [1] "Acurácia de teste do modelo com 30 neurônios: 80.8125 % ± 2.98777663330549 %"
## [1] "Acurácia de treinamento do modelo com 50 neurônios: 89.6315789473684 % ± 1.75059609117647 %"
## [1] "Acurácia de teste do modelo com 50 neurônios: 78.5625 % ± 3.97928435115768 %"
## [1] "Acurácia de treinamento do modelo com 100 neurônios: 96.3157894736842 % ± 1.3977258516945 %"
## [1] "Acurácia de teste do modelo com 100 neurônios: 71.9375 % ± 4.32014482578877 %"
## [1] "Acurácia de treinamento do modelo com 150 neurônios: 99.5526315789474 % ± 0.392189379142419 %"
## [1] "Acurácia de teste do modelo com 150 neurônios: 64.8125 % ± 7.35914888886079 %"
## [1] "Acurácia de treinamento do modelo com 200 neurônios: 100 % ± 0 %"
## [1] "Acurácia de teste do modelo com 200 neurônios: 57.5625 % ± 5.00780640600253 %"
## [1] "Acurácia de treinamento do modelo com 300 neurônios: 100 % ± 0 %"
## [1] "Acurácia de teste do modelo com 300 neurônios: 65.375 % ± 4.67882970072871 %"
\end{verbatim}

\hypertarget{perceptron-simples-com-base-de-dados-statlog-heart---dados-escalonados}{%
\paragraph{\texorpdfstring{\textbf{\emph{Perceptron simples com base de
dados Statlog (Heart) - Dados
escalonados}}}{Perceptron simples com base de dados Statlog (Heart) - Dados escalonados}}\label{perceptron-simples-com-base-de-dados-statlog-heart---dados-escalonados}}

\begin{Shaded}
\begin{Highlighting}[]
\KeywordTok{rm}\NormalTok{(}\DataTypeTok{list=}\KeywordTok{ls}\NormalTok{())}
\KeywordTok{source}\NormalTok{(}\StringTok{"~/Documents/UFMG/9/Redes Neurais/listas/lista 4/trainPerceptron.R"}\NormalTok{)}
\KeywordTok{source}\NormalTok{(}\StringTok{"~/Documents/UFMG/9/Redes Neurais/listas/lista 4/yperceptron.R"}\NormalTok{)}
\KeywordTok{source}\NormalTok{(}\StringTok{"~/Documents/UFMG/9/Redes Neurais/exemplos/escalonamento_matrix.R"}\NormalTok{)}
\KeywordTok{library}\NormalTok{(caret)}

\CommentTok{# Carregando base de dados:}
\NormalTok{path <-}\StringTok{ }\KeywordTok{file.path}\NormalTok{(}\StringTok{"~/Documents/UFMG/9/Redes Neurais/listas/lista 6/heart"}\NormalTok{, }\StringTok{"heart.csv"}\NormalTok{)}
\NormalTok{data <-}\StringTok{ }\KeywordTok{read.csv}\NormalTok{(path)}

\CommentTok{# Separando dados de entrada e saída:}
\NormalTok{x_all <-}\StringTok{ }\KeywordTok{as.matrix}\NormalTok{(data[}\DecValTok{1}\OperatorTok{:}\DecValTok{270}\NormalTok{, }\DecValTok{1}\OperatorTok{:}\DecValTok{13}\NormalTok{])}
\NormalTok{class <-}\StringTok{ }\KeywordTok{as.matrix}\NormalTok{(data[}\DecValTok{1}\OperatorTok{:}\DecValTok{270}\NormalTok{, }\DecValTok{14}\NormalTok{])}
\NormalTok{y_all <-}\StringTok{ }\KeywordTok{rep}\NormalTok{(}\DecValTok{0}\NormalTok{,}\DecValTok{270}\NormalTok{)}
\ControlFlowTok{for}\NormalTok{ (count }\ControlFlowTok{in} \DecValTok{1}\OperatorTok{:}\KeywordTok{length}\NormalTok{(class)) \{}
  \ControlFlowTok{if}\NormalTok{ (class[count] }\OperatorTok{==}\StringTok{ }\DecValTok{1}\NormalTok{ )\{}
\NormalTok{    y_all[count] =}\StringTok{ }\DecValTok{1}
\NormalTok{  \}}
  \ControlFlowTok{else} \ControlFlowTok{if}\NormalTok{(class[count] }\OperatorTok{==}\StringTok{ }\DecValTok{2}\NormalTok{)\{}
\NormalTok{    y_all[count] =}\StringTok{ }\DecValTok{0}
\NormalTok{  \}}
\NormalTok{\}}

\CommentTok{# Escalonando os valores dos atributos para que fiquem restritos entre 0 e 1}
\NormalTok{x_all <-}\StringTok{ }\KeywordTok{staggeringMatrix}\NormalTok{(x_all, }\KeywordTok{nrow}\NormalTok{(x_all), }\KeywordTok{ncol}\NormalTok{(x_all))}

\CommentTok{# Realiza pelo 20 execuções diferentes}
\NormalTok{accuracy_train <-}\StringTok{ }\KeywordTok{rep}\NormalTok{(}\DecValTok{0}\NormalTok{, }\DecValTok{20}\NormalTok{)}
\NormalTok{accuracy_test <-}\StringTok{ }\KeywordTok{rep}\NormalTok{(}\DecValTok{0}\NormalTok{, }\DecValTok{20}\NormalTok{)}
\ControlFlowTok{for}\NormalTok{(execution }\ControlFlowTok{in} \DecValTok{1}\OperatorTok{:}\DecValTok{20}\NormalTok{)\{}
  \CommentTok{# Separando dados entre treino e teste aleatoriamente:}
\NormalTok{  positions_train <-}\StringTok{ }\KeywordTok{createDataPartition}\NormalTok{(}\DecValTok{1}\OperatorTok{:}\DecValTok{270}\NormalTok{,}\DataTypeTok{p=}\NormalTok{.}\DecValTok{7}\NormalTok{)}
\NormalTok{  length_train <-}\StringTok{ }\KeywordTok{length}\NormalTok{(positions_train}\OperatorTok{$}\NormalTok{Resample1)}
\NormalTok{  length_test <-}\StringTok{ }\KeywordTok{length}\NormalTok{(y_all) }\OperatorTok{-}\StringTok{ }\NormalTok{length_train}
\NormalTok{  x_train <-}\StringTok{ }\KeywordTok{matrix}\NormalTok{(}\KeywordTok{rep}\NormalTok{(}\DecValTok{0}\NormalTok{, }\DecValTok{13}\OperatorTok{*}\NormalTok{length_train), }\DataTypeTok{ncol=}\DecValTok{13}\NormalTok{, }\DataTypeTok{nrow=}\NormalTok{length_train)}
\NormalTok{  y_train <-}\StringTok{ }\KeywordTok{rep}\NormalTok{(}\DecValTok{0}\NormalTok{, length_train)}
\NormalTok{  x_test <-}\StringTok{ }\KeywordTok{matrix}\NormalTok{(}\KeywordTok{rep}\NormalTok{(}\DecValTok{0}\NormalTok{, (}\DecValTok{13}\OperatorTok{*}\NormalTok{length_test)), }\DataTypeTok{ncol=}\DecValTok{13}\NormalTok{, }\DataTypeTok{nrow=}\NormalTok{(}\KeywordTok{length}\NormalTok{(y_all) }\OperatorTok{-}\StringTok{ }\NormalTok{length_train))}
\NormalTok{  y_test <-}\StringTok{ }\KeywordTok{rep}\NormalTok{(}\DecValTok{0}\NormalTok{, (}\KeywordTok{length}\NormalTok{(y_all) }\OperatorTok{-}\StringTok{ }\NormalTok{length_train))}
\NormalTok{  index_train <-}\StringTok{ }\DecValTok{1}
\NormalTok{  index_test <-}\StringTok{ }\DecValTok{1}
  \ControlFlowTok{for}\NormalTok{ (count }\ControlFlowTok{in} \DecValTok{1}\OperatorTok{:}\KeywordTok{length}\NormalTok{(y_all)) \{}
    \ControlFlowTok{if}\NormalTok{ (index_train }\OperatorTok{<=}\StringTok{ }\NormalTok{length_train }\OperatorTok{&&}\StringTok{ }\NormalTok{count }\OperatorTok{==}\StringTok{ }\NormalTok{positions_train}\OperatorTok{$}\NormalTok{Resample1[index_train])\{}
\NormalTok{      x_train[index_train, ] <-}\StringTok{ }\NormalTok{x_all[count, }\DecValTok{1}\OperatorTok{:}\DecValTok{13}\NormalTok{]}
\NormalTok{      y_train[index_train] <-}\StringTok{ }\NormalTok{y_all[count]}
\NormalTok{      index_train =}\StringTok{ }\NormalTok{index_train }\OperatorTok{+}\StringTok{ }\DecValTok{1}
\NormalTok{    \} }\ControlFlowTok{else}\NormalTok{ \{}
\NormalTok{      x_test[index_test, ] <-}\StringTok{ }\NormalTok{x_all[count, }\DecValTok{1}\OperatorTok{:}\DecValTok{13}\NormalTok{]}
\NormalTok{      y_test[index_test] <-}\StringTok{ }\NormalTok{y_all[count]}
\NormalTok{      index_test =}\StringTok{ }\NormalTok{index_test }\OperatorTok{+}\StringTok{ }\DecValTok{1}    
\NormalTok{    \}}
\NormalTok{  \}}
  
  \CommentTok{# Treinando modelo:}
\NormalTok{  retlist<-}\KeywordTok{trainPerceptron}\NormalTok{(x_train, y_train, }\FloatTok{0.1}\NormalTok{, }\FloatTok{0.01}\NormalTok{, }\DecValTok{1000}\NormalTok{, }\DecValTok{1}\NormalTok{)}
\NormalTok{  W<-retlist[[}\DecValTok{1}\NormalTok{]]}
  
  \CommentTok{# Calculando acurácia de treinamento}
\NormalTok{  y_hat_train <-}\StringTok{ }\KeywordTok{as.matrix}\NormalTok{(}\KeywordTok{yperceptron}\NormalTok{(x_train, W, }\DecValTok{1}\NormalTok{), }\DataTypeTok{nrow =}\NormalTok{ length_train, }\DataTypeTok{ncol =} \DecValTok{1}\NormalTok{)}
\NormalTok{  accuracy_train[execution]<-}\DecValTok{1}\OperatorTok{-}\NormalTok{((}\KeywordTok{t}\NormalTok{(y_hat_train}\OperatorTok{-}\NormalTok{y_train) }\OperatorTok\StringTok{ }\NormalTok{(y_hat_train}\OperatorTok{-}\NormalTok{y_train))}\OperatorTok{/}\NormalTok{length_train)}
  \CommentTok{#print(paste("Acurácia de treinamento para execução", execution, "com", p, "nerônios:", accuracy_train))}
  
  \CommentTok{# Calculando acurácia de Teste:}
\NormalTok{  y_hat_test <-}\StringTok{ }\KeywordTok{as.matrix}\NormalTok{(}\KeywordTok{yperceptron}\NormalTok{(x_test, W, }\DecValTok{1}\NormalTok{), }\DataTypeTok{nrow =}\NormalTok{ length_test, }\DataTypeTok{ncol =} \DecValTok{1}\NormalTok{)}
\NormalTok{  accuracy_test[execution]<-}\DecValTok{1}\OperatorTok{-}\NormalTok{((}\KeywordTok{t}\NormalTok{(y_hat_test}\OperatorTok{-}\NormalTok{y_test) }\OperatorTok\StringTok{ }\NormalTok{(y_hat_test}\OperatorTok{-}\NormalTok{y_test))}\OperatorTok{/}\NormalTok{length_test)}
  \CommentTok{#print(paste("Acurácia de teste para execução", execution, "com", p, "nerônios:", accuracy_test))}
\NormalTok{\}}
\CommentTok{# Média das acurácias}
\NormalTok{mean_accuracy_train <-}\StringTok{ }\KeywordTok{mean}\NormalTok{(accuracy_train) }\OperatorTok{*}\StringTok{ }\DecValTok{100}
\NormalTok{mean_accuracy_test <-}\StringTok{ }\KeywordTok{mean}\NormalTok{(accuracy_test) }\OperatorTok{*}\StringTok{ }\DecValTok{100}

\CommentTok{# Desvio Padrão das acurácias}
\NormalTok{sd_accuracy_train <-}\StringTok{ }\KeywordTok{sd}\NormalTok{(accuracy_train) }\OperatorTok{*}\StringTok{ }\DecValTok{100}
\NormalTok{sd_accuracy_test <-}\StringTok{ }\KeywordTok{sd}\NormalTok{(accuracy_test) }\OperatorTok{*}\StringTok{ }\DecValTok{100}

\KeywordTok{print}\NormalTok{(}\KeywordTok{paste}\NormalTok{(}\StringTok{"Acurácia de treinamento do modelo com perceptron simples"}\NormalTok{, mean_accuracy_train, }\StringTok{"%"}\NormalTok{, }\StringTok{"±"}\NormalTok{, sd_accuracy_train, }\StringTok{"%"}\NormalTok{))}
\end{Highlighting}
\end{Shaded}

\begin{verbatim}
## [1] "Acurácia de treinamento do modelo com perceptron simples 82.7105263157895 % ± 5.35181560879491 %"
\end{verbatim}

\begin{Shaded}
\begin{Highlighting}[]
\KeywordTok{print}\NormalTok{(}\KeywordTok{paste}\NormalTok{(}\StringTok{"Acurácia de teste do modelo com perceptron simples"}\NormalTok{, mean_accuracy_test, }\StringTok{"%"}\NormalTok{, }\StringTok{"±"}\NormalTok{, sd_accuracy_test, }\StringTok{"%"}\NormalTok{))}
\end{Highlighting}
\end{Shaded}

\begin{verbatim}
## [1] "Acurácia de teste do modelo com perceptron simples 79.1875 % ± 4.10140781349911 %"
\end{verbatim}

\hypertarget{discussuxe3o-4}{%
\paragraph{\texorpdfstring{\textbf{\emph{Discussão}}}{Discussão}}\label{discussuxe3o-4}}

Pôde-se perceber que escalonar os dados fez diferença no resultado da
acurácia de todos os modelos testados. Novamente, constatou-se que em
ambas as bases de dados o modelo com o perceptron simples se saiu melhor
que as ELMs, obtendo maiores valores acurácia.

\end{document}
